\usepackage{color}
\usepackage{amssymb}
\usepackage{amsmath}
\usepackage{amsthm}
\usepackage{amsfonts}
\usepackage{bm}

%

%--------------
%% preamble.tex
%% this should be included with a command like
%% \usepackage{color}
\usepackage{amssymb}
\usepackage{amsmath}
\usepackage{amsthm}
\usepackage{amsfonts}
\usepackage{bm}

%

%--------------
%% preamble.tex
%% this should be included with a command like
%% \usepackage{color}
\usepackage{amssymb}
\usepackage{amsmath}
\usepackage{amsthm}
\usepackage{amsfonts}
\usepackage{bm}

%

%--------------
%% preamble.tex
%% this should be included with a command like
%% \usepackage{color}
\usepackage{amssymb}
\usepackage{amsmath}
\usepackage{amsthm}
\usepackage{amsfonts}
\usepackage{bm}

%

%--------------
%% preamble.tex
%% this should be included with a command like
%% \input{preamble.tex}
%% \lecture{1}{September 4, 2004}{Ronitt Rubinfeld}{name
%%  of poor scribe}

\hbadness=10000
\vbadness=10000

\setlength{\oddsidemargin}{.25in}
\setlength{\evensidemargin}{.25in}
\setlength{\textwidth}{6in}
\setlength{\topmargin}{-0.4in}
\setlength{\textheight}{8.5in}

\setlength{\parindent}{0in}
\setlength{\parskip}{\medskipamount}

\newcommand{\handout}[5]{
   %\renewcommand{\thepage}{#1-\arabic{page}}
   \noindent
   \begin{center}
   \framebox{
      \vbox{
    \hbox to 5.78in { {\bf 6.896 Sublinear Time Algorithms}
     	 \hfill #2 }
       \vspace{4mm}
       \hbox to 5.78in { {\Large \hfill #5  \hfill} }
       \vspace{2mm}
       \hbox to 5.78in { {\it #3 \hfill #4} }
      }
   }
   \end{center}
   \vspace*{4mm}
}

\newcommand{\lecture}[4]{\handout{#1}{#2}{Lecturer:
#3}{Scribe: #4}{Lecture #1}}

\newcounter{exple}
\setcounter{exple}{0}
\newtheorem{theorem}{Theorem}[section]
\newtheorem{corollary}[theorem]{Corollary}
\newtheorem{lemma}[theorem]{Lemma}
\providecommand*{\lemmaautorefname}{Lemma}
\newtheorem{observation}[theorem]{Observation}
\newtheorem{proposition}[theorem]{Proposition}
\newtheorem{definition}[theorem]{Definition}
\newtheorem{conjecture}[theorem]{Conjecture}
\newtheorem{problem}[theorem]{Problem}
\newtheorem{example}[exple]{Example}
\newtheorem{claim}[theorem]{Claim}
\newtheorem{fact}[theorem]{Fact}
\newtheorem{assumption}[theorem]{Assumption}

%Some Environments
\newenvironment{Definition}{\begin{it-def} \rm}{\end{it-def}}
\newenvironment{Question}{\begin{it-ques} \rm}{\end{it-ques}}
\newenvironment{Remark}{\begin{trivlist}%
\item[\hskip\labelsep{\bf Remark.}]~}{\end{trivlist}}
\newenvironment{Observation}{\begin{trivlist}%
\item[\hskip\labelsep{\bf Observation.}]~}{\end{trivlist}}
\newenvironment{Example}{\begin{trivlist}%
\item[\hskip\labelsep{\bf Example:}]~}{\end{trivlist}}
\newenvironment{Property}{\begin{trivlist}%
\item[\hskip\labelsep{\bf Property.}]~}{\end{trivlist}}
\newenvironment{OpenProblem}{\begin{trivlist}%
\item[\hskip\labelsep{\bf Open Problem.}]~}{\end{trivlist}}
\newenvironment{ChallengeProblem}{\begin{trivlist}%
\item[\hskip\labelsep{\bf Challenge Problem.}]~}{\end{trivlist}}
\newenvironment{Conjecture}{\begin{trivlist}%
\item[\hskip\labelsep{\bf Conjecture.}]~}{\end{trivlist}}
\newenvironment{AppendixTheorem}[1]{\setcounter{Theorem}{\ref{#1}}\setcounter{Claim}{0} \begin{trivlist}%
\item[\hskip\labelsep{\bf Theorem \ref{#1}.}]~}{\end{trivlist}}
\newenvironment{AppendixLemma}[1]{\setcounter{Theorem}{\ref{#1}}\setcounter{Claim}{0} \begin{trivlist}%
\item[\hskip\labelsep{\bf Lemma \ref{#1}.}]~}{\end{trivlist}}
\newenvironment{AppendixProof}[1]{\setcounter{Theorem}{\ref{#1}}\setcounter{Claim}{0} \noindent {\bf Proof of Theorem \ref{#1}:}}
{\hspace*{\fill}$\Box$~~~~~\vspace{5mm} }
\newenvironment{AppendixLemmaProof}[1]{\setcounter{Theorem}{\ref{#1}}\setcounter{Claim}{0} \noindent {\bf Proof of Lemma \ref{#1}:}}
{\hspace*{\fill}$\Box$~~~~~\vspace{5mm} }
\newcounter{algo}
\setcounter{algo}{0}
\newenvironment{algorithm}[1] {
\stepcounter{algo}
\vspace{10pt}
\begin{minipage}{0.8\textwidth}
{\bf Algorithm \Roman{algo}.} #1 \\} {
\end{minipage}
\vspace{10pt}}




%Useful for recalling theorems. (hat-tip texexchange)
\makeatletter
\newtheorem*{rep@theorem}{\rep@title}
\newcommand{\newreptheorem}[2]{%
\newenvironment{rep#1}[1]{%
 \def\rep@title{#2 \ref{##1} (restated)}%
 \begin{rep@theorem}}%
 {\end{rep@theorem}}}
\makeatother

\newreptheorem{theorem}{Theorem}
\newreptheorem{lemma}{Lemma}
\newreptheorem{conjecture}{Conjecture}
\newreptheorem{problem}{Problem}






%\newcommand{\qed}{\rule{7pt}{7pt}}
\newcommand{\dis}{\mathop{\mbox{\rm d}}\nolimits}
\newcommand{\per}{\mathop{\mbox{\rm per}}\nolimits}
\newcommand{\area}{\mathop{\mbox{\rm area}}\nolimits}
\newcommand{\cw}{\mathop{\rm cw}\nolimits}
\newcommand{\ccw}{\mathop{\rm ccw}\nolimits}
\newcommand{\DIST}{\mathop{\mbox{\rm DIST}}\nolimits}
\newcommand{\OP}{\mathop{\mbox{\it OP}}\nolimits}
\newcommand{\OPprime}{\mathop{\mbox{\it OP}^{\,\prime}}\nolimits}
\newcommand{\ihat}{\hat{\imath}}
\newcommand{\jhat}{\hat{\jmath}}
\newcommand{\abs}[1]{\mathify{\left| #1 \right|}}
\newcommand{\row}{\mathop{\mbox{\it row}}\nolimits}
\newcommand{\lin}{\mathop{\mbox{\it Lin}}\nolimits}
\newcommand{\maj}{\mathop{\mbox{\it Maj}}\nolimits}
%\newenvironment{proof}{\noindent{\bf Proof}\hspace*{1em}}{\qed\bigskip}
\newenvironment{proof-sketch}{\noindent{\bf Sketch of Proof}\hspace*{1em}}{\qed\bigskip}
\newenvironment{proof-idea}{\noindent{\bf Proof Idea}\hspace*{1em}}{\qed\bigskip}
\newenvironment{proof-of-lemma}[1]{\noindent{\bf Proof of Lemma #1}\hspace*{1em}}{\qed\bigskip}
\newenvironment{proof-attempt}{\noindent{\bf Proof Attempt}\hspace*{1em}}{\qed\bigskip}
\newenvironment{proofof}[1]{\noindent{\bf Proof}
of #1:\hspace*{1em}}{\qed\bigskip}
\newenvironment{remark}{\noindent{\bf Remark}\hspace*{1em}}{\bigskip}

% \makeatletter
% \@addtoreset{figure}{section}
% \@addtoreset{table}{section}
% \@addtoreset{equation}{section}
% \makeatother

%\newcommand{\FOR}{{\bf for}}
%\newcommand{\TO}{{\bf to}}
%\newcommand{\DO}{{\bf do}}
%\newcommand{\WHILE}{{\bf while}}
%\newcommand{\AND}{{\bf and}}
%\newcommand{\IF}{{\bf if}}
%\newcommand{\THEN}{{\bf then}}
%\newcommand{\ELSE}{{\bf else}}

% \renewcommand{\thefigure}{\thesection.\arabic{figure}}
% \renewcommand{\thetable}{\thesection.\arabic{table}}
% \renewcommand{\theequation}{\thesection.\arabic{equation}}

\makeatletter
\def\fnum@figure{{\bf Figure \thefigure}}
\def\fnum@table{{\bf Table \thetable}}
\long\def\@mycaption#1[#2]#3{\addcontentsline{\csname
  ext@#1\endcsname}{#1}{\protect\numberline{\csname
  the#1\endcsname}{\ignorespaces #2}}\par
  \begingroup
    \@parboxrestore
    \small
    \@makecaption{\csname fnum@#1\endcsname}{\ignorespaces #3}\par
  \endgroup}
\def\mycaption{\refstepcounter\@captype \@dblarg{\@mycaption\@captype}}
\makeatother

\newcommand{\figcaption}[1]{\mycaption[]{#1}}
\newcommand{\tabcaption}[1]{\mycaption[]{#1}}
\newcommand{\head}[1]{\chapter[Lecture \##1]{}}
\newcommand{\mathify}[1]{\ifmmode{#1}\else\mbox{$#1$}\fi}
%\renewcommand{\Pr}[1]{\mathify{\mbox{Pr}\left[#1\right]}}
%\newcommand{\Exp}[1]{\mathify{\mbox{Exp}\left[#1\right]}}
\newcommand{\bigO}O
\newcommand{\set}[1]{\mathify{\left\{ #1 \right\}}}
\def\half{\frac{1}{2}}

% Coding theory addenda

\newcommand{\enc}{{\sf Enc}}
\newcommand{\dec}{{\sf Dec}}
\newcommand{\Var}{{\rm Var}}
\newcommand{\Z}{{\mathbb Z}}
\newcommand{\F}{{\mathbb F}}
\newcommand{\integers}{{\mathbb Z}^{\geq 0}}
\newcommand{\R}{{\mathbb R}}
\newcommand{\Q}{{\mathbb Q}}
\newcommand{\K}{{\mathbb K}}
\newcommand{\eqdef}{{\stackrel{\rm def}{=}}}
\newcommand{\from}{{\leftarrow}}
\newcommand{\vol}{{\rm Vol}}
\newcommand{\poly}{{\rm poly}}
\newcommand{\ip}[1]{{\langle #1 \rangle}}
\newcommand{\wt}{\mathop{\rm wt}}
\renewcommand{\vec}[1]{{\mathbf #1}}
\newcommand{\mspan}{{\rm span}}
\newcommand{\rs}{{\rm RS}}
\newcommand{\RM}{{\rm RM}}
\newcommand{\Had}{{\rm Had}}
\newcommand{\calc}{{\cal C}}
%\newcommand{\binom}[2]{{#1 \choose #2}}

\newcommand{\fig}[4]{
        \begin{figure}
        \setlength{\epsfysize}{#2}
        \vspace{3mm}
        \centerline{\epsfbox{#4}}
        \caption{#3} \label{#1}
        \end{figure}
        }

\newcommand{\ord}{{\rm ord}}

\providecommand{\norm}[1]{\lVert #1 \rVert}
\newcommand{\embed}{{\rm Embed}}
\newcommand{\qembed}{\mbox{$q$-Embed}}
\newcommand{\calh}{{\cal H}}
\newcommand{\lp}{{\rm LP}}
\newcommand{\veca}{\mathbf{a}}
\newcommand{\vecb}{\mathbf{b}}
\newcommand{\vecf}{\mathbf{f}}
\newcommand{\vecx}{\mathbf{x}}
\newcommand{\vecX}{\mathbf{X}}
\newcommand{\vecu}{\mathbf{u}}
\newcommand{\FF}{\mathbb{F}}
\newcommand{\CC}{\mathbb{C	}}


%% accented words
\newcommand{\Hastad}{H{\aa}stad }
\newcommand{\Godel}{G\"{o}del }
\newcommand{\Mobius}{M\"{o}bius }
\newcommand{\Gauss}{Gau{\ss} }
\newcommand{\naive}{na\"{\i}ve }
\newcommand{\Naive}{Na\"{\i}ve }
\newcommand{\grobner}{gr\"{o}bner }

\newcommand{\Det}{\mathsf{Det}}
\newcommand{\Perm}{\mathsf{Perm}}
\newcommand{\ESym}{\mathsf{Sym}}
\newcommand{\PSym}{\mathsf{Pow}}

\newcommand{\SPS}{\Sigma\Pi\Sigma}
\newcommand{\SPSP}{\mathrm{\Sigma \Pi \Sigma \Pi}}
\newcommand{\hSPSP}{{\Sigma \Pi \Sigma \Pi}^{\mathrm{[hom]}}}
\newcommand{\hmSPSP}{{\Sigma \Pi \Sigma \Pi}^{\mathrm{[hom,mul]}}}
%\newcommand{\SPowS}{{\Sigma{}^{\wedge}\Sigma}}
%\newcommand{\SPowSPowS}{{\Sigma{}^{\wedge}\Sigma{}^{\wedge}\Sigma}}
\newcommand{\mySPSP}[2]{\Sigma{\Pi}^{[#1]}\Sigma\Pi^{[#2]}}
\newcommand{\SES}{\Sigma {\wedge} \Sigma}
\newcommand{\SESES}{\Sigma {\wedge} \Sigma {\wedge} \Sigma}
\newcommand{\mySES}[1]{\Sigma {\wedge^{[#1]}} \Sigma}
\newcommand{\mySESES}[2]{\Sigma {\wedge^{[#1]}} \Sigma {\wedge^{[#2]}} \Sigma}


\newcommand{\inparen}[1]{\left(#1\right)}             %\inparen{x+y}  is (x+y)
\newcommand{\inbrace}[1]{\left\{#1\right\}}           %\inbrace{x+y}  is {x+y}
\newcommand{\insquar}[1]{\left[#1\right]}             %\insquar{x+y}  is [x+y]
\newcommand{\inangle}[1]{\left\langle#1\right\rangle} %\inangle{A}    is <A>


\newcommand{\setdef}[2]{\inbrace{{#1}\ : \ {#2}}}      % E.g: \setdef{x}{f(x) = 0}
\newcommand{\innerproduct}[2]{\left\langle{#1},{#2}\right\rangle} %\innerproduct{x}{y} is <x,y>.
\newcommand{\zo}{\inbrace{0,1}}                        % Well just something that is used often!
\newcommand{\parderiv}[2]{\frac{\partial #1}{\partial #2}}
\newcommand{\pderiv}[2]{\partial_{#2}\inparen{#1}}
\newcommand{\zof}[2]{\inbrace{0,1}^{#1}\longrightarrow \inbrace{0,1}^{#2}}
\newcommand{\defeq}{\stackrel{\rm def}{=}}


\newcommand{\dob}{{\bm{\partial}}}
\newcommand{\dodo}[2]{\frac{\partial{#1}}{\partial{#2}}}





%%%%%%%%% Some Editorial macros %%%%%%%%%%%%%%%%%%%%%%%%%%%%%
\definecolor{edcolor}{rgb}{0,0.8,0.3}
\newcommand{\authnote}[2]{{\color{blue}{$<<<${ \footnotesize #1 notes: #2}$>>>$}}}
%\newcommand{\Anote}[1]{\authnote{Ankit}{#1}}
%\newcommand{\Nnote}[1]{\authnote{Neeraj}{#1}}
%\newcommand{\Rnote}[1]{\authnote{Ramprasad}{#1}}
%\newcommand{\Pnote}[1]{\authnote{Pritish}{#1}}
\newcommand{\Anote}[1]{}
\newcommand{\Nnote}[1]{}
\newcommand{\Ynote}[1]{}

%% \lecture{1}{September 4, 2004}{Ronitt Rubinfeld}{name
%%  of poor scribe}

\hbadness=10000
\vbadness=10000

\setlength{\oddsidemargin}{.25in}
\setlength{\evensidemargin}{.25in}
\setlength{\textwidth}{6in}
\setlength{\topmargin}{-0.4in}
\setlength{\textheight}{8.5in}

\setlength{\parindent}{0in}
\setlength{\parskip}{\medskipamount}

\newcommand{\handout}[5]{
   %\renewcommand{\thepage}{#1-\arabic{page}}
   \noindent
   \begin{center}
   \framebox{
      \vbox{
    \hbox to 5.78in { {\bf 6.896 Sublinear Time Algorithms}
     	 \hfill #2 }
       \vspace{4mm}
       \hbox to 5.78in { {\Large \hfill #5  \hfill} }
       \vspace{2mm}
       \hbox to 5.78in { {\it #3 \hfill #4} }
      }
   }
   \end{center}
   \vspace*{4mm}
}

\newcommand{\lecture}[4]{\handout{#1}{#2}{Lecturer:
#3}{Scribe: #4}{Lecture #1}}

\newcounter{exple}
\setcounter{exple}{0}
\newtheorem{theorem}{Theorem}[section]
\newtheorem{corollary}[theorem]{Corollary}
\newtheorem{lemma}[theorem]{Lemma}
\providecommand*{\lemmaautorefname}{Lemma}
\newtheorem{observation}[theorem]{Observation}
\newtheorem{proposition}[theorem]{Proposition}
\newtheorem{definition}[theorem]{Definition}
\newtheorem{conjecture}[theorem]{Conjecture}
\newtheorem{problem}[theorem]{Problem}
\newtheorem{example}[exple]{Example}
\newtheorem{claim}[theorem]{Claim}
\newtheorem{fact}[theorem]{Fact}
\newtheorem{assumption}[theorem]{Assumption}

%Some Environments
\newenvironment{Definition}{\begin{it-def} \rm}{\end{it-def}}
\newenvironment{Question}{\begin{it-ques} \rm}{\end{it-ques}}
\newenvironment{Remark}{\begin{trivlist}%
\item[\hskip\labelsep{\bf Remark.}]~}{\end{trivlist}}
\newenvironment{Observation}{\begin{trivlist}%
\item[\hskip\labelsep{\bf Observation.}]~}{\end{trivlist}}
\newenvironment{Example}{\begin{trivlist}%
\item[\hskip\labelsep{\bf Example:}]~}{\end{trivlist}}
\newenvironment{Property}{\begin{trivlist}%
\item[\hskip\labelsep{\bf Property.}]~}{\end{trivlist}}
\newenvironment{OpenProblem}{\begin{trivlist}%
\item[\hskip\labelsep{\bf Open Problem.}]~}{\end{trivlist}}
\newenvironment{ChallengeProblem}{\begin{trivlist}%
\item[\hskip\labelsep{\bf Challenge Problem.}]~}{\end{trivlist}}
\newenvironment{Conjecture}{\begin{trivlist}%
\item[\hskip\labelsep{\bf Conjecture.}]~}{\end{trivlist}}
\newenvironment{AppendixTheorem}[1]{\setcounter{Theorem}{\ref{#1}}\setcounter{Claim}{0} \begin{trivlist}%
\item[\hskip\labelsep{\bf Theorem \ref{#1}.}]~}{\end{trivlist}}
\newenvironment{AppendixLemma}[1]{\setcounter{Theorem}{\ref{#1}}\setcounter{Claim}{0} \begin{trivlist}%
\item[\hskip\labelsep{\bf Lemma \ref{#1}.}]~}{\end{trivlist}}
\newenvironment{AppendixProof}[1]{\setcounter{Theorem}{\ref{#1}}\setcounter{Claim}{0} \noindent {\bf Proof of Theorem \ref{#1}:}}
{\hspace*{\fill}$\Box$~~~~~\vspace{5mm} }
\newenvironment{AppendixLemmaProof}[1]{\setcounter{Theorem}{\ref{#1}}\setcounter{Claim}{0} \noindent {\bf Proof of Lemma \ref{#1}:}}
{\hspace*{\fill}$\Box$~~~~~\vspace{5mm} }
\newcounter{algo}
\setcounter{algo}{0}
\newenvironment{algorithm}[1] {
\stepcounter{algo}
\vspace{10pt}
\begin{minipage}{0.8\textwidth}
{\bf Algorithm \Roman{algo}.} #1 \\} {
\end{minipage}
\vspace{10pt}}




%Useful for recalling theorems. (hat-tip texexchange)
\makeatletter
\newtheorem*{rep@theorem}{\rep@title}
\newcommand{\newreptheorem}[2]{%
\newenvironment{rep#1}[1]{%
 \def\rep@title{#2 \ref{##1} (restated)}%
 \begin{rep@theorem}}%
 {\end{rep@theorem}}}
\makeatother

\newreptheorem{theorem}{Theorem}
\newreptheorem{lemma}{Lemma}
\newreptheorem{conjecture}{Conjecture}
\newreptheorem{problem}{Problem}






%\newcommand{\qed}{\rule{7pt}{7pt}}
\newcommand{\dis}{\mathop{\mbox{\rm d}}\nolimits}
\newcommand{\per}{\mathop{\mbox{\rm per}}\nolimits}
\newcommand{\area}{\mathop{\mbox{\rm area}}\nolimits}
\newcommand{\cw}{\mathop{\rm cw}\nolimits}
\newcommand{\ccw}{\mathop{\rm ccw}\nolimits}
\newcommand{\DIST}{\mathop{\mbox{\rm DIST}}\nolimits}
\newcommand{\OP}{\mathop{\mbox{\it OP}}\nolimits}
\newcommand{\OPprime}{\mathop{\mbox{\it OP}^{\,\prime}}\nolimits}
\newcommand{\ihat}{\hat{\imath}}
\newcommand{\jhat}{\hat{\jmath}}
\newcommand{\abs}[1]{\mathify{\left| #1 \right|}}
\newcommand{\row}{\mathop{\mbox{\it row}}\nolimits}
\newcommand{\lin}{\mathop{\mbox{\it Lin}}\nolimits}
\newcommand{\maj}{\mathop{\mbox{\it Maj}}\nolimits}
%\newenvironment{proof}{\noindent{\bf Proof}\hspace*{1em}}{\qed\bigskip}
\newenvironment{proof-sketch}{\noindent{\bf Sketch of Proof}\hspace*{1em}}{\qed\bigskip}
\newenvironment{proof-idea}{\noindent{\bf Proof Idea}\hspace*{1em}}{\qed\bigskip}
\newenvironment{proof-of-lemma}[1]{\noindent{\bf Proof of Lemma #1}\hspace*{1em}}{\qed\bigskip}
\newenvironment{proof-attempt}{\noindent{\bf Proof Attempt}\hspace*{1em}}{\qed\bigskip}
\newenvironment{proofof}[1]{\noindent{\bf Proof}
of #1:\hspace*{1em}}{\qed\bigskip}
\newenvironment{remark}{\noindent{\bf Remark}\hspace*{1em}}{\bigskip}

% \makeatletter
% \@addtoreset{figure}{section}
% \@addtoreset{table}{section}
% \@addtoreset{equation}{section}
% \makeatother

%\newcommand{\FOR}{{\bf for}}
%\newcommand{\TO}{{\bf to}}
%\newcommand{\DO}{{\bf do}}
%\newcommand{\WHILE}{{\bf while}}
%\newcommand{\AND}{{\bf and}}
%\newcommand{\IF}{{\bf if}}
%\newcommand{\THEN}{{\bf then}}
%\newcommand{\ELSE}{{\bf else}}

% \renewcommand{\thefigure}{\thesection.\arabic{figure}}
% \renewcommand{\thetable}{\thesection.\arabic{table}}
% \renewcommand{\theequation}{\thesection.\arabic{equation}}

\makeatletter
\def\fnum@figure{{\bf Figure \thefigure}}
\def\fnum@table{{\bf Table \thetable}}
\long\def\@mycaption#1[#2]#3{\addcontentsline{\csname
  ext@#1\endcsname}{#1}{\protect\numberline{\csname
  the#1\endcsname}{\ignorespaces #2}}\par
  \begingroup
    \@parboxrestore
    \small
    \@makecaption{\csname fnum@#1\endcsname}{\ignorespaces #3}\par
  \endgroup}
\def\mycaption{\refstepcounter\@captype \@dblarg{\@mycaption\@captype}}
\makeatother

\newcommand{\figcaption}[1]{\mycaption[]{#1}}
\newcommand{\tabcaption}[1]{\mycaption[]{#1}}
\newcommand{\head}[1]{\chapter[Lecture \##1]{}}
\newcommand{\mathify}[1]{\ifmmode{#1}\else\mbox{$#1$}\fi}
%\renewcommand{\Pr}[1]{\mathify{\mbox{Pr}\left[#1\right]}}
%\newcommand{\Exp}[1]{\mathify{\mbox{Exp}\left[#1\right]}}
\newcommand{\bigO}O
\newcommand{\set}[1]{\mathify{\left\{ #1 \right\}}}
\def\half{\frac{1}{2}}

% Coding theory addenda

\newcommand{\enc}{{\sf Enc}}
\newcommand{\dec}{{\sf Dec}}
\newcommand{\Var}{{\rm Var}}
\newcommand{\Z}{{\mathbb Z}}
\newcommand{\F}{{\mathbb F}}
\newcommand{\integers}{{\mathbb Z}^{\geq 0}}
\newcommand{\R}{{\mathbb R}}
\newcommand{\Q}{{\mathbb Q}}
\newcommand{\K}{{\mathbb K}}
\newcommand{\eqdef}{{\stackrel{\rm def}{=}}}
\newcommand{\from}{{\leftarrow}}
\newcommand{\vol}{{\rm Vol}}
\newcommand{\poly}{{\rm poly}}
\newcommand{\ip}[1]{{\langle #1 \rangle}}
\newcommand{\wt}{\mathop{\rm wt}}
\renewcommand{\vec}[1]{{\mathbf #1}}
\newcommand{\mspan}{{\rm span}}
\newcommand{\rs}{{\rm RS}}
\newcommand{\RM}{{\rm RM}}
\newcommand{\Had}{{\rm Had}}
\newcommand{\calc}{{\cal C}}
%\newcommand{\binom}[2]{{#1 \choose #2}}

\newcommand{\fig}[4]{
        \begin{figure}
        \setlength{\epsfysize}{#2}
        \vspace{3mm}
        \centerline{\epsfbox{#4}}
        \caption{#3} \label{#1}
        \end{figure}
        }

\newcommand{\ord}{{\rm ord}}

\providecommand{\norm}[1]{\lVert #1 \rVert}
\newcommand{\embed}{{\rm Embed}}
\newcommand{\qembed}{\mbox{$q$-Embed}}
\newcommand{\calh}{{\cal H}}
\newcommand{\lp}{{\rm LP}}
\newcommand{\veca}{\mathbf{a}}
\newcommand{\vecb}{\mathbf{b}}
\newcommand{\vecf}{\mathbf{f}}
\newcommand{\vecx}{\mathbf{x}}
\newcommand{\vecX}{\mathbf{X}}
\newcommand{\vecu}{\mathbf{u}}
\newcommand{\FF}{\mathbb{F}}
\newcommand{\CC}{\mathbb{C	}}


%% accented words
\newcommand{\Hastad}{H{\aa}stad }
\newcommand{\Godel}{G\"{o}del }
\newcommand{\Mobius}{M\"{o}bius }
\newcommand{\Gauss}{Gau{\ss} }
\newcommand{\naive}{na\"{\i}ve }
\newcommand{\Naive}{Na\"{\i}ve }
\newcommand{\grobner}{gr\"{o}bner }

\newcommand{\Det}{\mathsf{Det}}
\newcommand{\Perm}{\mathsf{Perm}}
\newcommand{\ESym}{\mathsf{Sym}}
\newcommand{\PSym}{\mathsf{Pow}}

\newcommand{\SPS}{\Sigma\Pi\Sigma}
\newcommand{\SPSP}{\mathrm{\Sigma \Pi \Sigma \Pi}}
\newcommand{\hSPSP}{{\Sigma \Pi \Sigma \Pi}^{\mathrm{[hom]}}}
\newcommand{\hmSPSP}{{\Sigma \Pi \Sigma \Pi}^{\mathrm{[hom,mul]}}}
%\newcommand{\SPowS}{{\Sigma{}^{\wedge}\Sigma}}
%\newcommand{\SPowSPowS}{{\Sigma{}^{\wedge}\Sigma{}^{\wedge}\Sigma}}
\newcommand{\mySPSP}[2]{\Sigma{\Pi}^{[#1]}\Sigma\Pi^{[#2]}}
\newcommand{\SES}{\Sigma {\wedge} \Sigma}
\newcommand{\SESES}{\Sigma {\wedge} \Sigma {\wedge} \Sigma}
\newcommand{\mySES}[1]{\Sigma {\wedge^{[#1]}} \Sigma}
\newcommand{\mySESES}[2]{\Sigma {\wedge^{[#1]}} \Sigma {\wedge^{[#2]}} \Sigma}


\newcommand{\inparen}[1]{\left(#1\right)}             %\inparen{x+y}  is (x+y)
\newcommand{\inbrace}[1]{\left\{#1\right\}}           %\inbrace{x+y}  is {x+y}
\newcommand{\insquar}[1]{\left[#1\right]}             %\insquar{x+y}  is [x+y]
\newcommand{\inangle}[1]{\left\langle#1\right\rangle} %\inangle{A}    is <A>


\newcommand{\setdef}[2]{\inbrace{{#1}\ : \ {#2}}}      % E.g: \setdef{x}{f(x) = 0}
\newcommand{\innerproduct}[2]{\left\langle{#1},{#2}\right\rangle} %\innerproduct{x}{y} is <x,y>.
\newcommand{\zo}{\inbrace{0,1}}                        % Well just something that is used often!
\newcommand{\parderiv}[2]{\frac{\partial #1}{\partial #2}}
\newcommand{\pderiv}[2]{\partial_{#2}\inparen{#1}}
\newcommand{\zof}[2]{\inbrace{0,1}^{#1}\longrightarrow \inbrace{0,1}^{#2}}
\newcommand{\defeq}{\stackrel{\rm def}{=}}


\newcommand{\dob}{{\bm{\partial}}}
\newcommand{\dodo}[2]{\frac{\partial{#1}}{\partial{#2}}}





%%%%%%%%% Some Editorial macros %%%%%%%%%%%%%%%%%%%%%%%%%%%%%
\definecolor{edcolor}{rgb}{0,0.8,0.3}
\newcommand{\authnote}[2]{{\color{blue}{$<<<${ \footnotesize #1 notes: #2}$>>>$}}}
%\newcommand{\Anote}[1]{\authnote{Ankit}{#1}}
%\newcommand{\Nnote}[1]{\authnote{Neeraj}{#1}}
%\newcommand{\Rnote}[1]{\authnote{Ramprasad}{#1}}
%\newcommand{\Pnote}[1]{\authnote{Pritish}{#1}}
\newcommand{\Anote}[1]{}
\newcommand{\Nnote}[1]{}
\newcommand{\Ynote}[1]{}

%% \lecture{1}{September 4, 2004}{Ronitt Rubinfeld}{name
%%  of poor scribe}

\hbadness=10000
\vbadness=10000

\setlength{\oddsidemargin}{.25in}
\setlength{\evensidemargin}{.25in}
\setlength{\textwidth}{6in}
\setlength{\topmargin}{-0.4in}
\setlength{\textheight}{8.5in}

\setlength{\parindent}{0in}
\setlength{\parskip}{\medskipamount}

\newcommand{\handout}[5]{
   %\renewcommand{\thepage}{#1-\arabic{page}}
   \noindent
   \begin{center}
   \framebox{
      \vbox{
    \hbox to 5.78in { {\bf 6.896 Sublinear Time Algorithms}
     	 \hfill #2 }
       \vspace{4mm}
       \hbox to 5.78in { {\Large \hfill #5  \hfill} }
       \vspace{2mm}
       \hbox to 5.78in { {\it #3 \hfill #4} }
      }
   }
   \end{center}
   \vspace*{4mm}
}

\newcommand{\lecture}[4]{\handout{#1}{#2}{Lecturer:
#3}{Scribe: #4}{Lecture #1}}

\newcounter{exple}
\setcounter{exple}{0}
\newtheorem{theorem}{Theorem}[section]
\newtheorem{corollary}[theorem]{Corollary}
\newtheorem{lemma}[theorem]{Lemma}
\providecommand*{\lemmaautorefname}{Lemma}
\newtheorem{observation}[theorem]{Observation}
\newtheorem{proposition}[theorem]{Proposition}
\newtheorem{definition}[theorem]{Definition}
\newtheorem{conjecture}[theorem]{Conjecture}
\newtheorem{problem}[theorem]{Problem}
\newtheorem{example}[exple]{Example}
\newtheorem{claim}[theorem]{Claim}
\newtheorem{fact}[theorem]{Fact}
\newtheorem{assumption}[theorem]{Assumption}

%Some Environments
\newenvironment{Definition}{\begin{it-def} \rm}{\end{it-def}}
\newenvironment{Question}{\begin{it-ques} \rm}{\end{it-ques}}
\newenvironment{Remark}{\begin{trivlist}%
\item[\hskip\labelsep{\bf Remark.}]~}{\end{trivlist}}
\newenvironment{Observation}{\begin{trivlist}%
\item[\hskip\labelsep{\bf Observation.}]~}{\end{trivlist}}
\newenvironment{Example}{\begin{trivlist}%
\item[\hskip\labelsep{\bf Example:}]~}{\end{trivlist}}
\newenvironment{Property}{\begin{trivlist}%
\item[\hskip\labelsep{\bf Property.}]~}{\end{trivlist}}
\newenvironment{OpenProblem}{\begin{trivlist}%
\item[\hskip\labelsep{\bf Open Problem.}]~}{\end{trivlist}}
\newenvironment{ChallengeProblem}{\begin{trivlist}%
\item[\hskip\labelsep{\bf Challenge Problem.}]~}{\end{trivlist}}
\newenvironment{Conjecture}{\begin{trivlist}%
\item[\hskip\labelsep{\bf Conjecture.}]~}{\end{trivlist}}
\newenvironment{AppendixTheorem}[1]{\setcounter{Theorem}{\ref{#1}}\setcounter{Claim}{0} \begin{trivlist}%
\item[\hskip\labelsep{\bf Theorem \ref{#1}.}]~}{\end{trivlist}}
\newenvironment{AppendixLemma}[1]{\setcounter{Theorem}{\ref{#1}}\setcounter{Claim}{0} \begin{trivlist}%
\item[\hskip\labelsep{\bf Lemma \ref{#1}.}]~}{\end{trivlist}}
\newenvironment{AppendixProof}[1]{\setcounter{Theorem}{\ref{#1}}\setcounter{Claim}{0} \noindent {\bf Proof of Theorem \ref{#1}:}}
{\hspace*{\fill}$\Box$~~~~~\vspace{5mm} }
\newenvironment{AppendixLemmaProof}[1]{\setcounter{Theorem}{\ref{#1}}\setcounter{Claim}{0} \noindent {\bf Proof of Lemma \ref{#1}:}}
{\hspace*{\fill}$\Box$~~~~~\vspace{5mm} }
\newcounter{algo}
\setcounter{algo}{0}
\newenvironment{algorithm}[1] {
\stepcounter{algo}
\vspace{10pt}
\begin{minipage}{0.8\textwidth}
{\bf Algorithm \Roman{algo}.} #1 \\} {
\end{minipage}
\vspace{10pt}}




%Useful for recalling theorems. (hat-tip texexchange)
\makeatletter
\newtheorem*{rep@theorem}{\rep@title}
\newcommand{\newreptheorem}[2]{%
\newenvironment{rep#1}[1]{%
 \def\rep@title{#2 \ref{##1} (restated)}%
 \begin{rep@theorem}}%
 {\end{rep@theorem}}}
\makeatother

\newreptheorem{theorem}{Theorem}
\newreptheorem{lemma}{Lemma}
\newreptheorem{conjecture}{Conjecture}
\newreptheorem{problem}{Problem}






%\newcommand{\qed}{\rule{7pt}{7pt}}
\newcommand{\dis}{\mathop{\mbox{\rm d}}\nolimits}
\newcommand{\per}{\mathop{\mbox{\rm per}}\nolimits}
\newcommand{\area}{\mathop{\mbox{\rm area}}\nolimits}
\newcommand{\cw}{\mathop{\rm cw}\nolimits}
\newcommand{\ccw}{\mathop{\rm ccw}\nolimits}
\newcommand{\DIST}{\mathop{\mbox{\rm DIST}}\nolimits}
\newcommand{\OP}{\mathop{\mbox{\it OP}}\nolimits}
\newcommand{\OPprime}{\mathop{\mbox{\it OP}^{\,\prime}}\nolimits}
\newcommand{\ihat}{\hat{\imath}}
\newcommand{\jhat}{\hat{\jmath}}
\newcommand{\abs}[1]{\mathify{\left| #1 \right|}}
\newcommand{\row}{\mathop{\mbox{\it row}}\nolimits}
\newcommand{\lin}{\mathop{\mbox{\it Lin}}\nolimits}
\newcommand{\maj}{\mathop{\mbox{\it Maj}}\nolimits}
%\newenvironment{proof}{\noindent{\bf Proof}\hspace*{1em}}{\qed\bigskip}
\newenvironment{proof-sketch}{\noindent{\bf Sketch of Proof}\hspace*{1em}}{\qed\bigskip}
\newenvironment{proof-idea}{\noindent{\bf Proof Idea}\hspace*{1em}}{\qed\bigskip}
\newenvironment{proof-of-lemma}[1]{\noindent{\bf Proof of Lemma #1}\hspace*{1em}}{\qed\bigskip}
\newenvironment{proof-attempt}{\noindent{\bf Proof Attempt}\hspace*{1em}}{\qed\bigskip}
\newenvironment{proofof}[1]{\noindent{\bf Proof}
of #1:\hspace*{1em}}{\qed\bigskip}
\newenvironment{remark}{\noindent{\bf Remark}\hspace*{1em}}{\bigskip}

% \makeatletter
% \@addtoreset{figure}{section}
% \@addtoreset{table}{section}
% \@addtoreset{equation}{section}
% \makeatother

%\newcommand{\FOR}{{\bf for}}
%\newcommand{\TO}{{\bf to}}
%\newcommand{\DO}{{\bf do}}
%\newcommand{\WHILE}{{\bf while}}
%\newcommand{\AND}{{\bf and}}
%\newcommand{\IF}{{\bf if}}
%\newcommand{\THEN}{{\bf then}}
%\newcommand{\ELSE}{{\bf else}}

% \renewcommand{\thefigure}{\thesection.\arabic{figure}}
% \renewcommand{\thetable}{\thesection.\arabic{table}}
% \renewcommand{\theequation}{\thesection.\arabic{equation}}

\makeatletter
\def\fnum@figure{{\bf Figure \thefigure}}
\def\fnum@table{{\bf Table \thetable}}
\long\def\@mycaption#1[#2]#3{\addcontentsline{\csname
  ext@#1\endcsname}{#1}{\protect\numberline{\csname
  the#1\endcsname}{\ignorespaces #2}}\par
  \begingroup
    \@parboxrestore
    \small
    \@makecaption{\csname fnum@#1\endcsname}{\ignorespaces #3}\par
  \endgroup}
\def\mycaption{\refstepcounter\@captype \@dblarg{\@mycaption\@captype}}
\makeatother

\newcommand{\figcaption}[1]{\mycaption[]{#1}}
\newcommand{\tabcaption}[1]{\mycaption[]{#1}}
\newcommand{\head}[1]{\chapter[Lecture \##1]{}}
\newcommand{\mathify}[1]{\ifmmode{#1}\else\mbox{$#1$}\fi}
%\renewcommand{\Pr}[1]{\mathify{\mbox{Pr}\left[#1\right]}}
%\newcommand{\Exp}[1]{\mathify{\mbox{Exp}\left[#1\right]}}
\newcommand{\bigO}O
\newcommand{\set}[1]{\mathify{\left\{ #1 \right\}}}
\def\half{\frac{1}{2}}

% Coding theory addenda

\newcommand{\enc}{{\sf Enc}}
\newcommand{\dec}{{\sf Dec}}
\newcommand{\Var}{{\rm Var}}
\newcommand{\Z}{{\mathbb Z}}
\newcommand{\F}{{\mathbb F}}
\newcommand{\integers}{{\mathbb Z}^{\geq 0}}
\newcommand{\R}{{\mathbb R}}
\newcommand{\Q}{{\mathbb Q}}
\newcommand{\K}{{\mathbb K}}
\newcommand{\eqdef}{{\stackrel{\rm def}{=}}}
\newcommand{\from}{{\leftarrow}}
\newcommand{\vol}{{\rm Vol}}
\newcommand{\poly}{{\rm poly}}
\newcommand{\ip}[1]{{\langle #1 \rangle}}
\newcommand{\wt}{\mathop{\rm wt}}
\renewcommand{\vec}[1]{{\mathbf #1}}
\newcommand{\mspan}{{\rm span}}
\newcommand{\rs}{{\rm RS}}
\newcommand{\RM}{{\rm RM}}
\newcommand{\Had}{{\rm Had}}
\newcommand{\calc}{{\cal C}}
%\newcommand{\binom}[2]{{#1 \choose #2}}

\newcommand{\fig}[4]{
        \begin{figure}
        \setlength{\epsfysize}{#2}
        \vspace{3mm}
        \centerline{\epsfbox{#4}}
        \caption{#3} \label{#1}
        \end{figure}
        }

\newcommand{\ord}{{\rm ord}}

\providecommand{\norm}[1]{\lVert #1 \rVert}
\newcommand{\embed}{{\rm Embed}}
\newcommand{\qembed}{\mbox{$q$-Embed}}
\newcommand{\calh}{{\cal H}}
\newcommand{\lp}{{\rm LP}}
\newcommand{\veca}{\mathbf{a}}
\newcommand{\vecb}{\mathbf{b}}
\newcommand{\vecf}{\mathbf{f}}
\newcommand{\vecx}{\mathbf{x}}
\newcommand{\vecX}{\mathbf{X}}
\newcommand{\vecu}{\mathbf{u}}
\newcommand{\FF}{\mathbb{F}}
\newcommand{\CC}{\mathbb{C	}}


%% accented words
\newcommand{\Hastad}{H{\aa}stad }
\newcommand{\Godel}{G\"{o}del }
\newcommand{\Mobius}{M\"{o}bius }
\newcommand{\Gauss}{Gau{\ss} }
\newcommand{\naive}{na\"{\i}ve }
\newcommand{\Naive}{Na\"{\i}ve }
\newcommand{\grobner}{gr\"{o}bner }

\newcommand{\Det}{\mathsf{Det}}
\newcommand{\Perm}{\mathsf{Perm}}
\newcommand{\ESym}{\mathsf{Sym}}
\newcommand{\PSym}{\mathsf{Pow}}

\newcommand{\SPS}{\Sigma\Pi\Sigma}
\newcommand{\SPSP}{\mathrm{\Sigma \Pi \Sigma \Pi}}
\newcommand{\hSPSP}{{\Sigma \Pi \Sigma \Pi}^{\mathrm{[hom]}}}
\newcommand{\hmSPSP}{{\Sigma \Pi \Sigma \Pi}^{\mathrm{[hom,mul]}}}
%\newcommand{\SPowS}{{\Sigma{}^{\wedge}\Sigma}}
%\newcommand{\SPowSPowS}{{\Sigma{}^{\wedge}\Sigma{}^{\wedge}\Sigma}}
\newcommand{\mySPSP}[2]{\Sigma{\Pi}^{[#1]}\Sigma\Pi^{[#2]}}
\newcommand{\SES}{\Sigma {\wedge} \Sigma}
\newcommand{\SESES}{\Sigma {\wedge} \Sigma {\wedge} \Sigma}
\newcommand{\mySES}[1]{\Sigma {\wedge^{[#1]}} \Sigma}
\newcommand{\mySESES}[2]{\Sigma {\wedge^{[#1]}} \Sigma {\wedge^{[#2]}} \Sigma}


\newcommand{\inparen}[1]{\left(#1\right)}             %\inparen{x+y}  is (x+y)
\newcommand{\inbrace}[1]{\left\{#1\right\}}           %\inbrace{x+y}  is {x+y}
\newcommand{\insquar}[1]{\left[#1\right]}             %\insquar{x+y}  is [x+y]
\newcommand{\inangle}[1]{\left\langle#1\right\rangle} %\inangle{A}    is <A>


\newcommand{\setdef}[2]{\inbrace{{#1}\ : \ {#2}}}      % E.g: \setdef{x}{f(x) = 0}
\newcommand{\innerproduct}[2]{\left\langle{#1},{#2}\right\rangle} %\innerproduct{x}{y} is <x,y>.
\newcommand{\zo}{\inbrace{0,1}}                        % Well just something that is used often!
\newcommand{\parderiv}[2]{\frac{\partial #1}{\partial #2}}
\newcommand{\pderiv}[2]{\partial_{#2}\inparen{#1}}
\newcommand{\zof}[2]{\inbrace{0,1}^{#1}\longrightarrow \inbrace{0,1}^{#2}}
\newcommand{\defeq}{\stackrel{\rm def}{=}}


\newcommand{\dob}{{\bm{\partial}}}
\newcommand{\dodo}[2]{\frac{\partial{#1}}{\partial{#2}}}





%%%%%%%%% Some Editorial macros %%%%%%%%%%%%%%%%%%%%%%%%%%%%%
\definecolor{edcolor}{rgb}{0,0.8,0.3}
\newcommand{\authnote}[2]{{\color{blue}{$<<<${ \footnotesize #1 notes: #2}$>>>$}}}
%\newcommand{\Anote}[1]{\authnote{Ankit}{#1}}
%\newcommand{\Nnote}[1]{\authnote{Neeraj}{#1}}
%\newcommand{\Rnote}[1]{\authnote{Ramprasad}{#1}}
%\newcommand{\Pnote}[1]{\authnote{Pritish}{#1}}
\newcommand{\Anote}[1]{}
\newcommand{\Nnote}[1]{}
\newcommand{\Ynote}[1]{}

%% \lecture{1}{September 4, 2004}{Ronitt Rubinfeld}{name
%%  of poor scribe}

\hbadness=10000
\vbadness=10000

\setlength{\oddsidemargin}{.25in}
\setlength{\evensidemargin}{.25in}
\setlength{\textwidth}{6in}
\setlength{\topmargin}{-0.4in}
\setlength{\textheight}{8.5in}

\setlength{\parindent}{0in}
\setlength{\parskip}{\medskipamount}

\newcommand{\handout}[5]{
   %\renewcommand{\thepage}{#1-\arabic{page}}
   \noindent
   \begin{center}
   \framebox{
      \vbox{
    \hbox to 5.78in { {\bf 6.896 Sublinear Time Algorithms}
     	 \hfill #2 }
       \vspace{4mm}
       \hbox to 5.78in { {\Large \hfill #5  \hfill} }
       \vspace{2mm}
       \hbox to 5.78in { {\it #3 \hfill #4} }
      }
   }
   \end{center}
   \vspace*{4mm}
}

\newcommand{\lecture}[4]{\handout{#1}{#2}{Lecturer:
#3}{Scribe: #4}{Lecture #1}}

\newcounter{exple}
\setcounter{exple}{0}
\newtheorem{theorem}{Theorem}[section]
\newtheorem{corollary}[theorem]{Corollary}
\newtheorem{lemma}[theorem]{Lemma}
\providecommand*{\lemmaautorefname}{Lemma}
\newtheorem{observation}[theorem]{Observation}
\newtheorem{proposition}[theorem]{Proposition}
\newtheorem{definition}[theorem]{Definition}
\newtheorem{conjecture}[theorem]{Conjecture}
\newtheorem{problem}[theorem]{Problem}
\newtheorem{example}[exple]{Example}
\newtheorem{claim}[theorem]{Claim}
\newtheorem{fact}[theorem]{Fact}
\newtheorem{assumption}[theorem]{Assumption}

%Some Environments
\newenvironment{Definition}{\begin{it-def} \rm}{\end{it-def}}
\newenvironment{Question}{\begin{it-ques} \rm}{\end{it-ques}}
\newenvironment{Remark}{\begin{trivlist}%
\item[\hskip\labelsep{\bf Remark.}]~}{\end{trivlist}}
\newenvironment{Observation}{\begin{trivlist}%
\item[\hskip\labelsep{\bf Observation.}]~}{\end{trivlist}}
\newenvironment{Example}{\begin{trivlist}%
\item[\hskip\labelsep{\bf Example:}]~}{\end{trivlist}}
\newenvironment{Property}{\begin{trivlist}%
\item[\hskip\labelsep{\bf Property.}]~}{\end{trivlist}}
\newenvironment{OpenProblem}{\begin{trivlist}%
\item[\hskip\labelsep{\bf Open Problem.}]~}{\end{trivlist}}
\newenvironment{ChallengeProblem}{\begin{trivlist}%
\item[\hskip\labelsep{\bf Challenge Problem.}]~}{\end{trivlist}}
\newenvironment{Conjecture}{\begin{trivlist}%
\item[\hskip\labelsep{\bf Conjecture.}]~}{\end{trivlist}}
\newenvironment{AppendixTheorem}[1]{\setcounter{Theorem}{\ref{#1}}\setcounter{Claim}{0} \begin{trivlist}%
\item[\hskip\labelsep{\bf Theorem \ref{#1}.}]~}{\end{trivlist}}
\newenvironment{AppendixLemma}[1]{\setcounter{Theorem}{\ref{#1}}\setcounter{Claim}{0} \begin{trivlist}%
\item[\hskip\labelsep{\bf Lemma \ref{#1}.}]~}{\end{trivlist}}
\newenvironment{AppendixProof}[1]{\setcounter{Theorem}{\ref{#1}}\setcounter{Claim}{0} \noindent {\bf Proof of Theorem \ref{#1}:}}
{\hspace*{\fill}$\Box$~~~~~\vspace{5mm} }
\newenvironment{AppendixLemmaProof}[1]{\setcounter{Theorem}{\ref{#1}}\setcounter{Claim}{0} \noindent {\bf Proof of Lemma \ref{#1}:}}
{\hspace*{\fill}$\Box$~~~~~\vspace{5mm} }
\newcounter{algo}
\setcounter{algo}{0}
\newenvironment{algorithm}[1] {
\stepcounter{algo}
\vspace{10pt}
\begin{minipage}{0.8\textwidth}
{\bf Algorithm \Roman{algo}.} #1 \\} {
\end{minipage}
\vspace{10pt}}




%Useful for recalling theorems. (hat-tip texexchange)
\makeatletter
\newtheorem*{rep@theorem}{\rep@title}
\newcommand{\newreptheorem}[2]{%
\newenvironment{rep#1}[1]{%
 \def\rep@title{#2 \ref{##1} (restated)}%
 \begin{rep@theorem}}%
 {\end{rep@theorem}}}
\makeatother

\newreptheorem{theorem}{Theorem}
\newreptheorem{lemma}{Lemma}
\newreptheorem{conjecture}{Conjecture}
\newreptheorem{problem}{Problem}






%\newcommand{\qed}{\rule{7pt}{7pt}}
\newcommand{\dis}{\mathop{\mbox{\rm d}}\nolimits}
\newcommand{\per}{\mathop{\mbox{\rm per}}\nolimits}
\newcommand{\area}{\mathop{\mbox{\rm area}}\nolimits}
\newcommand{\cw}{\mathop{\rm cw}\nolimits}
\newcommand{\ccw}{\mathop{\rm ccw}\nolimits}
\newcommand{\DIST}{\mathop{\mbox{\rm DIST}}\nolimits}
\newcommand{\OP}{\mathop{\mbox{\it OP}}\nolimits}
\newcommand{\OPprime}{\mathop{\mbox{\it OP}^{\,\prime}}\nolimits}
\newcommand{\ihat}{\hat{\imath}}
\newcommand{\jhat}{\hat{\jmath}}
\newcommand{\abs}[1]{\mathify{\left| #1 \right|}}
\newcommand{\row}{\mathop{\mbox{\it row}}\nolimits}
\newcommand{\lin}{\mathop{\mbox{\it Lin}}\nolimits}
\newcommand{\maj}{\mathop{\mbox{\it Maj}}\nolimits}
%\newenvironment{proof}{\noindent{\bf Proof}\hspace*{1em}}{\qed\bigskip}
\newenvironment{proof-sketch}{\noindent{\bf Sketch of Proof}\hspace*{1em}}{\qed\bigskip}
\newenvironment{proof-idea}{\noindent{\bf Proof Idea}\hspace*{1em}}{\qed\bigskip}
\newenvironment{proof-of-lemma}[1]{\noindent{\bf Proof of Lemma #1}\hspace*{1em}}{\qed\bigskip}
\newenvironment{proof-attempt}{\noindent{\bf Proof Attempt}\hspace*{1em}}{\qed\bigskip}
\newenvironment{proofof}[1]{\noindent{\bf Proof}
of #1:\hspace*{1em}}{\qed\bigskip}
\newenvironment{remark}{\noindent{\bf Remark}\hspace*{1em}}{\bigskip}

% \makeatletter
% \@addtoreset{figure}{section}
% \@addtoreset{table}{section}
% \@addtoreset{equation}{section}
% \makeatother

%\newcommand{\FOR}{{\bf for}}
%\newcommand{\TO}{{\bf to}}
%\newcommand{\DO}{{\bf do}}
%\newcommand{\WHILE}{{\bf while}}
%\newcommand{\AND}{{\bf and}}
%\newcommand{\IF}{{\bf if}}
%\newcommand{\THEN}{{\bf then}}
%\newcommand{\ELSE}{{\bf else}}

% \renewcommand{\thefigure}{\thesection.\arabic{figure}}
% \renewcommand{\thetable}{\thesection.\arabic{table}}
% \renewcommand{\theequation}{\thesection.\arabic{equation}}

\makeatletter
\def\fnum@figure{{\bf Figure \thefigure}}
\def\fnum@table{{\bf Table \thetable}}
\long\def\@mycaption#1[#2]#3{\addcontentsline{\csname
  ext@#1\endcsname}{#1}{\protect\numberline{\csname
  the#1\endcsname}{\ignorespaces #2}}\par
  \begingroup
    \@parboxrestore
    \small
    \@makecaption{\csname fnum@#1\endcsname}{\ignorespaces #3}\par
  \endgroup}
\def\mycaption{\refstepcounter\@captype \@dblarg{\@mycaption\@captype}}
\makeatother

\newcommand{\figcaption}[1]{\mycaption[]{#1}}
\newcommand{\tabcaption}[1]{\mycaption[]{#1}}
\newcommand{\head}[1]{\chapter[Lecture \##1]{}}
\newcommand{\mathify}[1]{\ifmmode{#1}\else\mbox{$#1$}\fi}
%\renewcommand{\Pr}[1]{\mathify{\mbox{Pr}\left[#1\right]}}
%\newcommand{\Exp}[1]{\mathify{\mbox{Exp}\left[#1\right]}}
\newcommand{\bigO}O
\newcommand{\set}[1]{\mathify{\left\{ #1 \right\}}}
\def\half{\frac{1}{2}}

% Coding theory addenda

\newcommand{\enc}{{\sf Enc}}
\newcommand{\dec}{{\sf Dec}}
\newcommand{\Var}{{\rm Var}}
\newcommand{\Z}{{\mathbb Z}}
\newcommand{\F}{{\mathbb F}}
\newcommand{\integers}{{\mathbb Z}^{\geq 0}}
\newcommand{\R}{{\mathbb R}}
\newcommand{\Q}{{\mathbb Q}}
\newcommand{\K}{{\mathbb K}}
\newcommand{\eqdef}{{\stackrel{\rm def}{=}}}
\newcommand{\from}{{\leftarrow}}
\newcommand{\vol}{{\rm Vol}}
\newcommand{\poly}{{\rm poly}}
\newcommand{\ip}[1]{{\langle #1 \rangle}}
\newcommand{\wt}{\mathop{\rm wt}}
\renewcommand{\vec}[1]{{\mathbf #1}}
\newcommand{\mspan}{{\rm span}}
\newcommand{\rs}{{\rm RS}}
\newcommand{\RM}{{\rm RM}}
\newcommand{\Had}{{\rm Had}}
\newcommand{\calc}{{\cal C}}
%\newcommand{\binom}[2]{{#1 \choose #2}}

\newcommand{\fig}[4]{
        \begin{figure}
        \setlength{\epsfysize}{#2}
        \vspace{3mm}
        \centerline{\epsfbox{#4}}
        \caption{#3} \label{#1}
        \end{figure}
        }

\newcommand{\ord}{{\rm ord}}

\providecommand{\norm}[1]{\lVert #1 \rVert}
\newcommand{\embed}{{\rm Embed}}
\newcommand{\qembed}{\mbox{$q$-Embed}}
\newcommand{\calh}{{\cal H}}
\newcommand{\lp}{{\rm LP}}
\newcommand{\veca}{\mathbf{a}}
\newcommand{\vecb}{\mathbf{b}}
\newcommand{\vecf}{\mathbf{f}}
\newcommand{\vecx}{\mathbf{x}}
\newcommand{\vecX}{\mathbf{X}}
\newcommand{\vecu}{\mathbf{u}}
\newcommand{\FF}{\mathbb{F}}
\newcommand{\CC}{\mathbb{C	}}


%% accented words
\newcommand{\Hastad}{H{\aa}stad }
\newcommand{\Godel}{G\"{o}del }
\newcommand{\Mobius}{M\"{o}bius }
\newcommand{\Gauss}{Gau{\ss} }
\newcommand{\naive}{na\"{\i}ve }
\newcommand{\Naive}{Na\"{\i}ve }
\newcommand{\grobner}{gr\"{o}bner }

\newcommand{\Det}{\mathsf{Det}}
\newcommand{\Perm}{\mathsf{Perm}}
\newcommand{\ESym}{\mathsf{Sym}}
\newcommand{\PSym}{\mathsf{Pow}}

\newcommand{\SPS}{\Sigma\Pi\Sigma}
\newcommand{\SPSP}{\mathrm{\Sigma \Pi \Sigma \Pi}}
\newcommand{\hSPSP}{{\Sigma \Pi \Sigma \Pi}^{\mathrm{[hom]}}}
\newcommand{\hmSPSP}{{\Sigma \Pi \Sigma \Pi}^{\mathrm{[hom,mul]}}}
%\newcommand{\SPowS}{{\Sigma{}^{\wedge}\Sigma}}
%\newcommand{\SPowSPowS}{{\Sigma{}^{\wedge}\Sigma{}^{\wedge}\Sigma}}
\newcommand{\mySPSP}[2]{\Sigma{\Pi}^{[#1]}\Sigma\Pi^{[#2]}}
\newcommand{\SES}{\Sigma {\wedge} \Sigma}
\newcommand{\SESES}{\Sigma {\wedge} \Sigma {\wedge} \Sigma}
\newcommand{\mySES}[1]{\Sigma {\wedge^{[#1]}} \Sigma}
\newcommand{\mySESES}[2]{\Sigma {\wedge^{[#1]}} \Sigma {\wedge^{[#2]}} \Sigma}


\newcommand{\inparen}[1]{\left(#1\right)}             %\inparen{x+y}  is (x+y)
\newcommand{\inbrace}[1]{\left\{#1\right\}}           %\inbrace{x+y}  is {x+y}
\newcommand{\insquar}[1]{\left[#1\right]}             %\insquar{x+y}  is [x+y]
\newcommand{\inangle}[1]{\left\langle#1\right\rangle} %\inangle{A}    is <A>


\newcommand{\setdef}[2]{\inbrace{{#1}\ : \ {#2}}}      % E.g: \setdef{x}{f(x) = 0}
\newcommand{\innerproduct}[2]{\left\langle{#1},{#2}\right\rangle} %\innerproduct{x}{y} is <x,y>.
\newcommand{\zo}{\inbrace{0,1}}                        % Well just something that is used often!
\newcommand{\parderiv}[2]{\frac{\partial #1}{\partial #2}}
\newcommand{\pderiv}[2]{\partial_{#2}\inparen{#1}}
\newcommand{\zof}[2]{\inbrace{0,1}^{#1}\longrightarrow \inbrace{0,1}^{#2}}
\newcommand{\defeq}{\stackrel{\rm def}{=}}


\newcommand{\dob}{{\bm{\partial}}}
\newcommand{\dodo}[2]{\frac{\partial{#1}}{\partial{#2}}}





%%%%%%%%% Some Editorial macros %%%%%%%%%%%%%%%%%%%%%%%%%%%%%
\definecolor{edcolor}{rgb}{0,0.8,0.3}
\newcommand{\authnote}[2]{{\color{blue}{$<<<${ \footnotesize #1 notes: #2}$>>>$}}}
%\newcommand{\Anote}[1]{\authnote{Ankit}{#1}}
%\newcommand{\Nnote}[1]{\authnote{Neeraj}{#1}}
%\newcommand{\Rnote}[1]{\authnote{Ramprasad}{#1}}
%\newcommand{\Pnote}[1]{\authnote{Pritish}{#1}}
\newcommand{\Anote}[1]{}
\newcommand{\Nnote}[1]{}
\newcommand{\Ynote}[1]{}
