\section{Some simple lower bounds}

Let us start with the simplest complete\footnote{in the sense that any polynomial can be computed in this model albeit of large size}  class of arithmetic circuits -- depth-$2$ circuits or $\Sigma\Pi$ circuits. 

\subsection{Lower bounds for $\Sigma\Pi$ circuits}

Any $\Sigma\Pi$ circuit of size $s$ computes a polynomial $f = m_1 + \dots + m_s$ where each $m_i$ is a monomial multiplied by a field constant. Therefore, any polynomial computed by a \emph{small} $\Sigma\Pi$ circuit must have a \emph{small} number of monomials. Hence, it is obvious that any polynomial that has many monomials require large $\Sigma\Pi$ circuits. 

This can be readily rephrased in the language of the outline described last section by defining $\Gamma(f)$ to simply be the number of monomials present in $f$. Hence, $\Gamma(f)\leq s$ for any $f$ computed by a $\Sigma\Pi$ circuit of size $s$. Of course, even a polynomial like $f = (x_1 + x_2+\dots + x_n)^n$  have $\Gamma(f) = n^{\Omega(n)}$ giving the lower bound. 

\subsection{Lower bounds for $\Sigma\!\wedge\!\Sigma$ circuits}

A $\Sigma\!\wedge\!\Sigma$ circuit of size $s$ computes a polynomial of the form $f = \ell_1^{d_1} + \dots + \ell_s^{d_s}$ where each $\ell_i$ is a linear polynomial over the $n$ variables.\footnote{such circuits are also called \emph{diagonal depth-$3$ circuits} in the literature}

Clearly as even a single $\ell^d$ could have exponentially many monomials, the $\Gamma$ defined above cannot work in this setting. Nevertheless, we shall try to design a similar map to ensure that $\Gamma(f)$ is \emph{small} whenever $f$ is computable by a \emph{small} $\Sigma\!\wedge\!\Sigma$ circuit. \\

In this setting, the \emph{building blocks} are terms of the form $\ell^d$. The goal would be to construct a \emph{sub-additive} measure $\Gamma$ such that $\Gamma(\ell^d)$ is \emph{small}. Here is the key observation to guide us towards a good choice of $\Gamma$. 

\begin{observation}
Any $k$-th order partial derivative of $\ell^d$ is a constant multiple of $\ell^{d-k}$. 
\end{observation}

Hence, if $\partial^{=k}(f)$ denotes the set of $k$-th order partial derivatives of $f$, then the space spanned by $\partial^{=k}(\ell^d)$ has dimension $1$. This naturally leads us to define $\Gamma$ exploiting this weakness. 

$$
\Gamma_k(f)\quad \eqdef \quad \dim \inparen{\partial^{=k}(f)}
$$

It is straightforward to check that $\Gamma_k$ is indeed sub-additive and hence $\Gamma_k(f) \leq s$ whenever $f$ is computable by a $\Sigma\!\wedge\!\Sigma$ circuit of size $s$. For a random polynomial $f$, we should be expecting $\Gamma_k(f)$ to be $\binom{n+k}{k}$ as there is unlikely to be any linear dependencies among the partial derivatives. Hence, all that needs to be done is to find an explicit polynomial with large $\Gamma_k$. 


If we consider $\Det_n$ or $\Perm_n$, then any partial derivative of order $k$ is just an $(n-k)\times(n-k)$ minor. Also, these minors consist of disjoint sets of monomials and hence are linearly independent. Hence, $\Gamma_k(\Det_n) = \binom{n}{k}^2$. Choosing $k = n/2$, we immediately get that any $\Sigma\!\wedge\!\Sigma$ circuit computing $\Det_n$ or $\Perm_n$ must be of size $2^{\Omega(n)}$. \\

\subsection{Low-rank $\Sigma\Pi\Sigma$}\label{sec:low-rank-sps}

A slight generalization of $\Sigma\!\wedge\!\Sigma$ circuits is a \emph{rank-$r$ $\Sigma\Pi\Sigma$ circuit} that computes a polynomial of the form 
$$
f \spaced{=}  T_1 \;+\; \dots \;+\; T_s
$$
where each $T_i = \ell_{i1}\dots \ell_{id}$ is a product of linear polynomials such that $\dim\inbrace{\ell_{i1},\dots, \ell_{id}}\leq r$. \\

Thus, $\Sigma\!\wedge\!\Sigma$  is a rank-$1$ $\Sigma\Pi\Sigma$ circuit, and a similar partial-derivative technique for lower bounds works here as well. 

In the setting where $r$ is much smaller than the number of variables $n$, each $T_i$ is essentially an $r$-variate polynomial masquerading as an $n$-variate polynomial using an affine transformation. In particular, the set of $n$ first order derivatives of $T$ have rank at most $r$. This yields the following observation.

\begin{observation}
Let $T = \ell_1\dots \ell_d$ with $\dim\inbrace{\ell_1,\dots, \ell_d}\leq r$. Then for any $k$, we have
$$
\Gamma_k(T) \spaced{\eqdef}\dim\inparen{\partial^{=k}(T)} \spaced{\leq} \binom{r+k}{k}.
$$
\end{observation}

Thus once again by sub-additivity, for any polynomial $f$ computable by a rank-$r$ $\Sigma\Pi\Sigma$ circuit of size $s$, we have $\Gamma_k(f) \leq s\cdot \binom{r+k}{r}$. Note that a random polynomial is expected to have $\Gamma_k(f)$ close to $\binom{n+k}{k}$, which could be much larger for $r\ll n$. We already saw that $\Gamma_k(\Det_n) = \binom{n}{k}^2$. This immediately gives the following lower bound, the proof of which we leave as an exercise to the interested reader. 

\begin{theorem}\label{thm:low-rank-sps-lb}
Let $r \leq n^{2-\delta}$ for some constant $\delta > 0$. For $k = \epsilon n^{\delta}$, where $\epsilon > 0$ is sufficiently small, we have
$$
\frac{\binom{n}{k}^2}{\binom{r+k}{k}} \quad=\quad \exp\inparen{\Omega(n^{\delta})}.
$$
Hence, any rank-$r$ $\Sigma\Pi\Sigma$ circuit computing $\Det_n$ or $\Perm_n$ must have size $\exp\inparen{\Omega(n^\delta)}$. \qed
\end{theorem}


This technique of using the rank of partial derivatives was introduced by Nisan and Wigderson \cite{nw1997} to prove lower bounds for \emph{homogeneous depth-$3$ circuits} (which also follows as a corollary of Theorem~\ref{thm:low-rank-sps-lb}). The survey of Chen, Kayal and Wigderson \cite{ckw11} give a comprehensive exposition of the power of the \emph{partial derivative method}. \\



With these simple examples, we can move on to other lower bounds for various other more interesting models. 



%%% Local Variables: 
%%% mode: latex
%%% TeX-master: "lowerbounds_birk"
%%% End: 
