\section{Weak lower bounds for general circuits and formulas}\label{sec:gen-ckt-formulas}

Despite several attempts by various researchers to prove lower bounds for arithmetic circuits or formulas, we only have very mild lower bounds for general circuits or formulas thus far. In this section, we shall look at the two  modest lower bounds for general circuits and formulas. 

\subsection{Lower bounds for general circuits}\label{sec:baur-strassen}

The only super-linear lower bound we currently know for general arithmetic circuits is the following  result of Baur and Strassen \cite{BS83}.

\begin{theorem}[\cite{BS83}]\label{thm:baur-strassen}
  Any fan-in $2$ circuit that computes the polynomial $f = x_1^{d+1} + \dots + x_n^{d+1}$ has size $\Omega(n\log d)$. 
\end{theorem}

\subsubsection{An exploitable weakness}

Each gate of the circuit $\Phi$ computes a local operation on the two children. To formalize this, define a new variable $y_g$ for every gate $g \in \Phi$. Further, for every gate $g$ define a quadratic equation $Q_g$ as
$$
Q_g = \begin{cases} y_g - (y_{g_1} + y_{g_2}) & \text{if $g = g_1 + g_2$}\\
  y_g - (y_{g_1}\cdot y_{g_2}) & \text{if $g = g_1 \cdot g_2$}.
\end{cases}
$$
Further if $y_o$  corresponds to the output gate, then the system of equations
$$\setdef{Q_g = 0}{g\in \Phi} \spaced{\union} \inbrace{y_{o} = 1}$$
completely characterize the computations of $\Phi$ that results in an output of $1$. 

The same can also be extended for \emph{multi-output} circuits that compute several polynomials simultaneously. In such cases, the set of equations
$$\setdef{Q_g = 0}{g\in \Phi} \spaced{\union} \setdef{y_{o_i} = 1}{i=1, \ldots, n}$$
completely characterize computations that result in an output of all ones. The following classical theorem allows us to bound the number of  common roots to a system of polynomial equations. 

\begin{theorem}[\Bezout's theorem]
  Let $g_1,\dots, g_r \in \F[X]$ such that $\deg(g_i) = d_i$ such that the number of common roots of $g_1=\dots=g_r = 0$ is finite. Then, the number of common roots (counted with multiplicities) is bounded by $\prod d_i$.
\end{theorem}

Thus in particular, if we have a circuit $\Phi$ of size $s$ that \emph{simultaneously} computes $\inbrace{x_1^d, \dots,x_n^d}$, then we have $d^n$ inputs that evaluate to all ones (where each $x_i$ must be  a $d$-th root of unity). Hence, \Bezout's theorem implies that
$$
2^s\spaced{\geq} d^n \spaced{\quad\implies\quad} s \spaced{=} \Omega(d\log n).
$$

Observe that $\inbrace{x_1^d,\dots, x_n^d}$ are all first-order derivatives of $f = x_1^{d+1}+\dots+x_n^{d+1}$ (with suitable scaling). A natural question here is the following --- if $f$ can be computed an arithmetic circuit of size $s$, what is the size required to compute all first-order partial derivatives of $f$ simultaneously? The \naive approach of computing each derivative separately results in a circuit of size $O(s\cdot n)$. Baur and Strassen \cite{BS83} show that we can save a factor of $n$.

\begin{lemma}[\cite{BS83}]\label{lem:baur-strassen}
  Let $\Phi$ be an arithmetic circuit of size $s$ and fan-in $2$ that computes a polynomial $f\in \F[X]$. Then, there is a multi-output circuit  of size $O(s)$ computing all first order derivatives of $f$.
\end{lemma}

Note that this immediately implies that any circuit computing $f = x_1^{d+1} + \dots + x_n^{d+1}$ requires size $\Omega(d\log n)$ as claimed by Theorem~\ref{thm:baur-strassen}. 


\subsubsection{Computing all first order derivatives simultaneously}

Since we are working with fan-in $2$ circuits, the number of edges is at most twice the size. Hence let $s$ denote the number of edges in the circuit $\Phi$, and we shall prove by induction that all first order derivatives of $\Phi$ can be computed by a circuit of size at most $5s$. Pick a non-leaf node $v$ in the circuit $\Phi$ closest to the leaves with both its children being variables, and say $x_1$ and $x_2$ are the variables feeding into $v$. In other words, $v = x_1 \odot x_2$ where $\odot$ is either $+$ or $\times$.

Let $\Phi'$ be the circuit obtained by deleting the two edges feeding into $v$, and replacing $v$ by a new variable. Hence, $\Phi'$ computes a polynomial $f' \in \F[X\union \inbrace{v}]$ and has at most $(s-1)$ edges. By induction on the size, we can assume that there is a circuit $\mathbb{D}(\Phi')$ consisting of at most $5(s-1)$ edges that computes all the first order derivatives of $f'$.

Observe that since $f'\mid_{(v = x_1 \odot x_2)} = f(\vecx)$,  we have that 
$$
\parderiv{f}{x_i} \spaced{=}\inparen{\parderiv{f'}{x_i}}_{v = x_1 \odot x_2} \quad+\quad  \inparen{\parderiv{f'}{v}}_{v = x_1 \odot x_2}\inparen{\parderiv{(x_1 \odot x_2)}{x_i}}.
$$

Hence, if $v = x_1 + x_2$ then
\begin{eqnarray*}
  \parderiv{f}{x_1} & = & \inparen{\parderiv{f'}{x_1}}_{v=x_1 + x_2} +\quad \inparen{\parderiv{f'}{v}}_{v = x_1 + x_2}\\
  \parderiv{f}{x_2} & = & \inparen{\parderiv{f'}{x_2}}_{v=x_1 + x_2} +\quad \inparen{\parderiv{f'}{v}}_{v = x_1 + x_2}\\
  \parderiv{f}{x_i} & = & \inparen{\parderiv{f'}{x_i}}_{v=x_1 + x_2} \qquad\text{for $i>2$}.
\end{eqnarray*}
If $v = x_1 \cdot x_2$, then
\begin{eqnarray*}
  \parderiv{f}{x_1} & = & \inparen{\parderiv{f'}{x_1}}_{v=x_1 \cdot x_2} + \inparen{\parderiv{f'}{v}}_{v = x_1 \cdot x_2} \cdot x_2\\
  \parderiv{f}{x_2} & = & \inparen{\parderiv{f'}{x_2}}_{v=x_1 \cdot x_2} + \inparen{\parderiv{f'}{v}}_{v = x_1\cdot x_2}\cdot x_1\\
  \parderiv{f}{x_i} & = & \inparen{\parderiv{f'}{x_i}}_{v=x_1 \cdot x_2} \qquad\text{for $i>2$}.
\end{eqnarray*}

Hence, by adding at most $5$ additional edges to $\mathbb{D}(\Phi')$, we can construct $\mathbb{D}(\Phi)$ and hence size of $\mathbb{D}(\Phi)$ is at most $5s$. \qed (Lemma~\ref{lem:baur-strassen})

\subsection{Lower bounds for formulas}\label{sec:Kalorkoti}

This section would be devoted to the proof of Kalorkoti's lower bound
\cite{k85} for formulas computing $\Det_n$, $\Perm_n$.

\begin{theorem}[\cite{k85}]\label{thm:kalorkoti}
  Any arithmetic formula computing $\Perm_n$ (or $\Det_n$) requires
  $\Omega(n^3)$ size.
\end{theorem}

The exploitable weakness in this setting is again to use the fact that the polynomials computed at intermediate gates share many polynomial dependencies. 

\begin{definition}[Algebraic independence]
  A set of polynomials $\inbrace{f_1,\dots, f_m}$ is said to be
  \emph{algebraically independent} if there is no non-trivial polynomial 
  $H(z_1,\dots, z_m)$ such that $H(f_1,\dots, f_m)=0$. 

  The size of the largest algebraically independent subset of
  $\vecf=\inbrace{f_1,\dots, f_m}$ is called the \emph{transcendence
    degree} (denoted by $\mathrm{trdeg}(f)$).
\end{definition}

The proof of Kalorkoti's theorem proceeds by defining a \emph{complexity measure} using the above notion of algebraic independence. \\


{\bf The Measure:} 
For any subset of variables $Y\subseteq X$, we can write a polynomial
$f \in \F[X]$ of the form $f = \sum_{i=1}^s f_i \cdot m_i$ where $m_i$'s are
distinct monomials in the variables in $Y$, and $f_i \in
F[X \setminus Y]$. We shall denote by $\mathrm{td}_Y(f)$ the transcendence degree of $\inbrace{f_1,\dots, f_s}$


Fix a partition of variables $X = X_1 \sqcup \dots
\sqcup X_r$. For any polynomial $f\in \F[X]$, define the map $\CM{Kal}:\F[X]\rightarrow \Z_{\geq 0}$  as
$$
\CM{Kal}(f) \quad=\quad \sum_{i=1}^r \mathrm{td}_{X_i}(f).
$$

The lower bound proceeds in two natural steps:
\begin{enumerate}
\item Show that $\CM{Kal}(f)$ is \emph{small} whenever $f$ is computable by a \emph{small} formula. 
\item Show that $\CM{Kal}(\Det_n)$ is \emph{large}. 
\end{enumerate}

\subsubsection{Upper bounding $\CM{Kal}$ for a formula}

\begin{lemma}\label{lem:kal-upperbound}
  Let $f$ be computed by a fan-in two formula $\Phi$ of size $s$. Then
  for any partition of variables $X = X_1\sqcup \dots \sqcup X_r$, we
  have $\CM{Kal}(f) = O(s)$.
\end{lemma}
\begin{proof}
  For any node $v\in \Phi$, let $\textsc{Leaf}(v)$ denote the leaves
  of the subtree rooted at $v$ and let $\textsc{Leaf}_{X_i}(v)$ denote
  the leaves of the subtree rooted at $v$ that are in the part
  $X_i$. Since the underlying graph of $\Phi$ is a tree, it follows
  that the size of $\Phi$ is bounded by a twice the number of
  leaves. For each part $X_i$, we shall show that
  $\mathrm{td}_{X_i}(f) = O(\abs{\textsc{Leaf}_{X_i}(\Phi)})$, which
  would prove the required bound. \\

  Fix an arbitrary part $Y = X_i$. Define the following three 
  sets of nodes
  \begin{eqnarray*}
    V_0 & = & \setdef{v\in \Phi}{\abs{\textsc{Leaf}_{Y}(v)} = 0 \quad\text{and}\quad \abs{\textsc{Leaf}_{Y}(\textsc{Parent}(v))} \geq 2}\\
    V_1 & = & \setdef{v\in \Phi}{\abs{\textsc{Leaf}_{Y}(v)} = 1 \quad\text{and}\quad \abs{\textsc{Leaf}_{Y}(\textsc{Parent}(v))} \geq 2}\\
    V_2 & = & \setdef{v\in \Phi}{\abs{\textsc{Leaf}_{Y}(v)} \geq 2}.
  \end{eqnarray*}

  Each node $v\in V_0$ computes a polynomial in $f_v \in
  \F[X\setminus Y]$, and we shall replace the subtree at $v$ by a node
  computing the polynomial $f_v$. Similarly, any node $v\in V_1$
  computes a polynomial of the form $f^{(0)}_v + f^{(1)}_v y_v$ for some $y_v\in Y$
  and $f^{(0)}_v, f^{(1)}_v \in \F[X\setminus Y]$. We shall again replace the
  subtree rooted at $v$ by a node computing $f^{(0)}_v + f^{(1)}_v y_v$. 

  Hence, the formula $\Phi$ now reduces to a smaller formula $\Phi_Y$ with
  leaves  being the nodes in $V_0$ and $V_1$ (and nodes in $V_2$ are
  unaffected). We would like to show that the size of the reduced
  formula, which is at most twice the number of its leaves, is
  $O(\abs{\textsc{Leaf}_Y(\Phi)})$.

  \begin{observation}\label{obs:v1-bound}$\abs{V_1} \leq \abs{\textsc{Leaf}_Y(\Phi)}$.    
  \end{observation}
  \begin{myproof}{Obs}
    Each node in $V_1$ has a distinct leaf labelled with a variable in
    $Y$. Hence, $\abs{V_1}$ is bounded by the number of leaves
    labelled with a variable in $Y$.
  \end{myproof}
  
  This shows that the $V_1$ leaves are not too many. Unfortunately, we
  cannot immediately bound the number of $V_0$ leaves, since we could
  have a long chain of $V_2$ nodes each with one sibling being a $V_0$
  leaf. The following observation would show how we can eliminate such
  long chains.

  \begin{observation}\label{obs:same-leaf-collapse}
    Let $u$ be an arbitrary node, and $v$ be another node in the
    subtree rooted at $u$ with $\textsc{Leaf}_Y(u) =
    \textsc{Leaf}_Y(v)$. Then the polynomial $g_u$ computed at $u$ and
    the polynomial $g_v$ computed at $v$ are related as $g_u = f_1 g_v
    + f_2$ for some $f_1,f_2 \in \F[X\setminus Y]$.
  \end{observation}
  \begin{myproof}{Obs}
    If $\textsc{Leaf}_Y(u) =\textsc{Leaf}_Y(v)$, then every node on
    the path from $u$ to $v$ must have a $V_0$ leaf as the other child. The
    observation follows as all these nodes are $+$ or $\times$ gates.
  \end{myproof}

  Using the above observation, we shall remove the need for $V_0$
  nodes completely by augmenting each node $u$ (computing a polynomial
  $g_u$) in $\Phi_Y$ with polynomials $f_u^{(0)}, f_u^{(1)} \in \F[X\setminus Y]$
  to enable them to compute $f_u^{(1)}g_u + f_u^{(0)}$. Let this augmented formula be called $\hat{\Phi}_Y$. Using
  Observation~\ref{obs:same-leaf-collapse}, we can now contract any
  two nodes $u$ and $v$ with $\textsc{Leaf}_Y(u) =
  \textsc{Leaf}_Y(v)$, and eliminate all $V_0$ nodes
  completely. Since all $V_2$ nodes are internal nodes, the only leaves of the augmented formula $\hat{\Phi}_Y$ are in $V_1$. Hence, the size of the augmented formula $\hat{\Phi}_Y$ is  bounded by $2\abs{V_1}$, which is bounded by
  $2\abs{\textsc{Leaf}_Y(\Phi)}$ by Observation~\ref{obs:v1-bound}.\\

  Suppose $\Phi$ computes a polynomial $f$, which can be written as  $f
  = \sum_{i=1}^t f_i\cdot m_i$ with $f_i \in \F[X\setminus Y]$ and $m_i$'s being
  distinct monomials in $Y$. Since $\hat{\Phi}_Y$ also computes $f$, each $f_i$ is a polynomial
  combination of the polynomials $S_Y = \setdef{f_{u}^{(0)},
    f_{u}^{(1)}}{u\in \hat{\Phi}_Y}$. Since $\hat{\Phi}_Y$ consists of at
  most $2\abs{\textsc{Leaf}_Y(\Phi)}$ augmented nodes, we have that
  $\mathrm{td}_Y(f) \leq |S_Y| \leq 4\abs{\textsc{Leaf}_Y(\Phi)}$. Therefore, 
  $$
  \mathrm{td}_Y(f) \quad=\quad \mathrm{trdeg}\setdef{f_i}{i\in [t]} \quad\leq\quad 4\abs{\textsc{Leaf}_Y(\Phi)}
  $$
  Hence, 
  $$\CM{Kal}(\Phi) = \sum_{i=1}^r \mathrm{td}_{X_i}(f_i) \leq 4\inparen{\sum_{i=1}^r \abs{\textsc{Leaf}_{X_i}}} = O(s).
  $$
\end{proof}

\subsubsection{Lower bounding $\CM{Kal}(\Det_n)$}


\begin{lemma}\label{lem:kal-lowerbound}
  Let $X = X_1 \sqcup \dots \sqcup X_n$ be the partition as defined by
  $X_t = \setdef{x_{ij}}{i-j\equiv t\bmod{n}}$. Then,
  $\CM{Kal}(\Det_n) = \Omega(n^3)$.
\end{lemma}
\begin{proof}
  By symmetry, it is easy to see that $\text{td}_{X_i}(\Det_n)$ is
  the same for all $i$. Hence, it suffices to show that
  $\text{td}_{Y}(\Det_n) = \Omega(n^2)$ for $Y = X_n = \inbrace{x_{11},\dots, x_{nn}}$. 

  To see this, observe that the determinant consists of the monomials
  $\inparen{\frac{x_{11}\dots x_{nn}}{x_{ii}x_{jj}}}\cdot
  x_{ij}x_{ji}$ for every $i\neq j$. Hence, $\text{td}_{Y}(\Det_n)
  \geq \mathrm{trdeg}\setdef{x_{ij}x_{ji}}{i\neq j} =
  \Omega(n^2)$. Therefore, $\CM{Kal}(\Det_n) =
  \Omega(n^3)$.
\end{proof}

The proof of Theorem~\ref{thm:kalorkoti} follows from
Lemma~\ref{lem:kal-upperbound} and Lemma~\ref{lem:kal-lowerbound}.


%%% Local Variables: 
%%% mode: latex
%%% TeX-master: "lowerbounds_birk"
%%% End: 



%%% LOCAL Variables: 
%%% mode: latex
%%% TeX-master: "lowerbounds"
%%% End: 
