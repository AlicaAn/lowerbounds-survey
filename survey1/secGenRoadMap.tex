\section{``Natural'' proof strategies}\label{sec:roadmap}

The lower bounds presented in Section~\ref{sec:gen-ckt-formulas} proceeded by first identifying a \emph{weakness} of the model, and exploiting it in an explicit manner. More concretely, Section~\ref{sec:Kalorkoti} presents a promising strategy that could be adopted to prove lower bounds for various models of arithmetic circuits. The crux of the lower bound was the construction of a good map $\Gamma$ that assigned a number to every polynomial. The map $\CM{Kal}$ was useful to show a lower bound in the sense that any $f$ computable by a \emph{small} formula had \emph{small} $\CM{Kal}(f)$. In fact, all subsequent lower bounds in arithmetic circuit complexity have more or less followed a similar template of a ``natural proof''. More concretely, all the subsequent lower bounds we shall see would essentially follow the outlined plan.  

\begin{quote}
{\bf Step 1 (normal forms)} For every circuit in the circuit class $\mathcal{C}$ of interest, express the polynomial computed as a \emph{small sum of simple building blocks}. 
\end{quote}

For example, every $\Sigma\Pi\Sigma$ circuit is a \emph{small} sum of \emph{products of linear polynomials} which are the building blocks here. In this case, the circuit model naturally admits such a representation but we shall see other examples with very different representations as sum of building blocks. 

\begin{quote}
{\bf Step 2 (complexity measure)} Construct a map $\Gamma: \F[x_1,\dots, x_n] \rightarrow \Z_{\geq 0}$ that is \emph{sub-additive} i.e. $\Gamma(f_1 + f_2)\leq \Gamma(f_1) + \Gamma(f_2)$.
\end{quote}

In most cases, $\Gamma(f)$ is the rank of a large matrix whose entries are linear functions in the coefficients of $f$. In such cases, we immediately get that $\Gamma$ is sub-additive. 

The strength of the choice of $\Gamma$ is determined by the next step. 

\begin{quote}
{\bf Step 3 (potential usefulness)} Show that if $B$ is a \emph{simple building block}, then $\Gamma(B)$ is \emph{small}.
Further, check if $\Gamma(f)$ for a \emph{random polynomial} $f$ is large (potentially). 
\end{quote}

This would suggest that if any $f$ with large $\Gamma(f)$ is to be written as a sum of $B_1 + \dots + B_s$, then sub-additivity and the fact that $\Gamma(B_i)$ is small for each $i$ and $\Gamma(f)$ is large immediately imply that $s$ must be large. This implies that the complexity measure $\Gamma$ does indeed have a potential to prove a lower bound for the class. The next step is just to replace the \emph{random polynomial} by an explicit polynomial. 

\begin{quote}
{\bf Step 4 (explicit lower bound)} Find an explicit polynomial $f$ for which $\Gamma(f)$ is large. 
\end{quote} 



These are usually the steps taken in almost all the known arithmetic circuit lower bound proofs. The main ingenuity lies in constructing a useful complexity measure, which is really to design $\Gamma$ so that it is small on the \emph{building blocks}. \\

Of course, there could potentially be lower bound proofs that do not follow the road-map outlined. For instance, it could be possible that $\Gamma$ is not small for a random polynomial, but specifically tailored in a way to make $\Gamma$ large for the $\Perm_n$. Or perhaps $\Gamma$ need not even be sub-additive and maybe there is a very different way to argue that all polynomial in the circuit class have small $\Gamma$. However, this has been the road-map for almost all lower bounds so far (barring very few exceptions). As a warmup, we first present some very simple applications of the above plan to prove lower bounds for some very simple subclasses of arithmetic circuits in the next section. We then move on to more sophisticated proofs of lower bounds for less restricted subclasses of circuits. 


%%% Local Variables: 
%%% mode: latex
%%% TeX-master: "lowerbounds"
%%% End: 
