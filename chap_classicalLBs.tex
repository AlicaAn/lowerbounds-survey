\chapter{Lower bounds for general circuits and formulas}\label{chap:gen-ckt-formulas}

Despite several attempts by various researchers to prove lower bounds for arithmetic circuits or formulas, we only have very mild lower bounds for general circuits or formulas thus far. 
In this chapter, we shall look at the two  modest lower bounds for general circuits and formulas. 

\section{Lower bounds for general circuits}\label{sec:baur-strassen}

The only super-linear lower bound we currently know for general arithmetic circuits is the following  result of Baur and Strassen \cite{BS83}.

\begin{theorem}[\cite{BS83}]\label{thm:baur-strassen}
  Any fan-in $2$ circuit that computes the polynomial $f = x_1^{d+1} + \dots + x_n^{d+1}$ has $\Omega(n\log d)$ edges. 
\end{theorem}


\subsection{An exploitable weakness}

Without loss of generality, let us assume that the circuit is a fan-in $2$ circuit. 
This would allow us to talk in terms of the number of nodes instead of edges. 

Each gate of the circuit $\Phi$ computes a local operation on the two children. 
To formalize this, define a new variable $y_g$ for every gate $g \in \Phi$. 
Further, for every gate $g$ define a quadratic equation $Q_g$ as
$$
Q_g = \begin{cases} y_g - (y_{g_1} + y_{g_2}) & \text{if $g = g_1 + g_2$}\\
  y_g - (y_{g_1}\cdot y_{g_2}) & \text{if $g = g_1 \cdot g_2$}.
\end{cases}
$$
Further if $y_o$  corresponds to the output gate, then the system of equations
$$\setdef{Q_g = 0}{g\in \Phi} \spaced{\union} \inbrace{y_{o} = 1}$$
completely characterize the computations of $\Phi$ that results in an output of $1$. 

The same can also be extended for \emph{multi-output} circuits that compute several polynomials simultaneously. 
In such cases, the set of equations
$$\setdef{Q_g = 0}{g\in \Phi} \spaced{\union} \setdef{y_{o_i} = 1}{i=1, \ldots, n}$$
completely characterize computations that result in an output of all ones. 
The following classical theorem allows us to bound the number of  common roots to a system of polynomial equations. 

\begin{theorem}[\Bezout's theorem]
  Let $g_1,\dots, g_r \in \F[X]$ such that $\deg(g_i) = d_i$ such that the number of common roots of $g_1=\dots=g_r = 0$ is finite. 
Then, the number of common roots (counted with multiplicities) is bounded by $\prod d_i$.
\end{theorem}

Thus in particular, if we have a circuit $\Phi$ of size $s$ that \emph{simultaneously} computes $\inbrace{x_1^d, \dots,x_n^d}$, then we have $d^n$ inputs that evaluate to all ones (where each $x_i$ must be  a $d$-th root of unity). 
Hence, \Bezout's theorem implies that
$$
2^s\spaced{\geq} d^n \spaced{\quad\implies\quad} s \spaced{=} \Omega(d\log n).
$$

Observe that $\inbrace{x_1^d,\dots, x_n^d}$ are all first-order derivatives of $f = x_1^{d+1}+\dots+x_n^{d+1}$ (with suitable scaling). 
A natural question here is the following --- if $f$ can be computed an arithmetic circuit of size $s$, what is the size required to compute all first-order partial derivatives of $f$ simultaneously? 
The \naive approach of computing each derivative separately results in a circuit of size $O(s\cdot n)$. 
Baur and Strassen \cite{BS83} show that we can save a factor of $n$.

\begin{lemma}[\cite{BS83}]\label{lem:baur-strassen}
  Let $\Phi$ be an arithmetic circuit of size $s$ and fan-in $2$ that computes a polynomial $f\in \F[X]$. 
Then, there is a multi-output circuit  of size $O(s)$ computing all first order derivatives of $f$.
\end{lemma}

Note that this immediately implies that any circuit computing $f = x_1^{d+1} + \dots + x_n^{d+1}$ requires size $\Omega(d\log n)$ as claimed by \autoref{thm:baur-strassen}. 


\subsection{Computing all first order derivatives simultaneously}

Since we are working with fan-in $2$ circuits, the number of edges is at most twice the size. 
Hence let $s$ denote the number of edges in the circuit $\Phi$, and we shall prove by induction that all first order derivatives of $\Phi$ can be computed by a circuit of size at most $5s$. 
Pick a non-leaf node $v$ in the circuit $\Phi$ closest to the leaves with both its children being variables, and say $x_1$ and $x_2$ are the variables feeding into $v$. 
In other words, $v = x_1 \odot x_2$ where $\odot$ is either $+$ or $\times$.

Let $\Phi'$ be the circuit obtained by deleting the two edges feeding into $v$, and replacing $v$ by a new variable. 
Hence, $\Phi'$ computes a polynomial $f' \in \F[X\union \inbrace{v}]$ and has at most $(s-2)$ edges. 
By induction on the size, we can assume that there is a circuit $\mathbb{D}(\Phi')$ consisting of at most $5(s-2)$ edges that computes all the first order derivatives of $f'$.

Observe that since $f'\mid_{(v = x_1 \odot x_2)} = f(\vecx)$,  we have that 
$$
\parderiv{f}{x_i} \spaced{=}\inparen{\parderiv{f'}{x_i}}_{v = x_1 \odot x_2} \quad+\quad  \inparen{\parderiv{f'}{v}}_{v = x_1 \odot x_2}\inparen{\parderiv{(x_1 \odot x_2)}{x_i}}.
$$

Hence, if $v = x_1 + x_2$ then
\begin{eqnarray*}
  \parderiv{f}{x_1} & = & \inparen{\parderiv{f'}{x_1}}_{v=x_1 + x_2} +\quad \inparen{\parderiv{f'}{v}}_{v = x_1 + x_2}\\
  \parderiv{f}{x_2} & = & \inparen{\parderiv{f'}{x_2}}_{v=x_1 + x_2} +\quad \inparen{\parderiv{f'}{v}}_{v = x_1 + x_2}\\
  \parderiv{f}{x_i} & = & \inparen{\parderiv{f'}{x_i}}_{v=x_1 + x_2} \qquad\text{for $i>2$}.
\end{eqnarray*}
If $v = x_1 \cdot x_2$, then
\begin{eqnarray*}
  \parderiv{f}{x_1} & = & \inparen{\parderiv{f'}{x_1}}_{v=x_1 \cdot x_2} + \inparen{\parderiv{f'}{v}}_{v = x_1 \cdot x_2} \cdot x_2\\
  \parderiv{f}{x_2} & = & \inparen{\parderiv{f'}{x_2}}_{v=x_1 \cdot x_2} + \inparen{\parderiv{f'}{v}}_{v = x_1\cdot x_2}\cdot x_1\\
  \parderiv{f}{x_i} & = & \inparen{\parderiv{f'}{x_i}}_{v=x_1 \cdot x_2} \qquad\text{for $i>2$}.
\end{eqnarray*}

Hence, by adding at most $10$ additional edges (see \autoref{fig:baur-strassen}, with the additional edges marked in red) to $\mathbb{D}(\Phi')$, we can construct $\mathbb{D}(\Phi)$ with at most $5s$ edges. \qed (\autoref{lem:baur-strassen})

\begin{figure}[h]
\begin{center}
\tikzstyle{gate}=[circle,draw=black!50,thick,inner sep=0]
\tikzstyle{leaf}=[circle,draw=black!50,fill=black!10,thick,inner sep=0]
\begin{tikzpicture}
  \draw[fill=gray!20] (1.5,4) -- ++(2,0) -- ++(-1,1) -- cycle;
  \node at (2.5,4.3) {$f$};
  \node[gate] (g) at (1.5,3.5) {\tiny $\times$}
  edge[-*] (1.9,4.2);
  \node at (2.5,4.3) {$f$};
  \node[leaf] (x1) at (1.2,3) {\tiny $x_1$}
  edge[->] (g);
  \node[leaf] (x2) at (1.8,3) {\tiny $x_2$}
  edge[->] (g);
  \node[text=gray] at (2.5,2.5) {$s$ edges};

  \draw[fill=gray!20] (8.5,4) -- ++(2,0) -- ++(-1,1) -- cycle;
  \node at (9.5,4.3) {$f'$};
  \node[leaf] (v) at (8.5,3.5) {\tiny $\;v\;$}
  edge[-*] (8.9,4.2);
  \node[text=gray] at (9.5,2.5) {$(s-2)$ edges};

  \draw[fill=gray!20] (0,-3) -- ++(5,0) -- ++(-1,1) -- ++(-3,0) -- cycle;
  \node[leaf] (x11) at (0.9,-4.3) {\tiny $x_1$}
  edge[-*] (0.9,-2.8);
  \node[leaf] (x12) at (1.5,-4.3) {\tiny $x_2$}
  edge[-*] (1.5,-2.8);
  \node[gate] at (0.3,-3.5) {\tiny $\times$}
  edge[-*] (0.3,-2.8)
  edge[<-,red] (x11)
  edge[<-,red] (x12);

  \path (1.1,-2.2) edge[thin] ++(0,0.4);
  \node[gate] (m1) at (0.8,-1.4) {\tiny $\times$}
  edge[<-,red] (1.1,-1.8)
  edge[<-,red,bend right=20] (x11)  ;
  \node[gate] (m2) at (1.4,-1.4) {\tiny $\times$}
  edge[<-,red] (1.1,-1.8)
  edge[<-,red,bend left=20] (x12)  ;

  \path (1.8,-2.2) edge[thin] ++(0,0.4);
  \node[gate] (s1) at (1.8,-0.5) {\tiny $+$}
  edge[<-] (1.8,-1.8)
  edge[<-,red] (m2)
  edge[-,red] (1.8,0);
  \node at (1.8,0.3) {\tiny $\partial_1 f$};

  \path (2.5,-2.2) edge[thin] ++(0,0.4);
  \node[gate] (s2) at (2.5,-0.5) {\tiny $+$}
  edge[<-] (2.5,-1.8)
  edge[<-,red] (m1)
  edge[-,red] (2.5,0);
  \node at (2.5,0.3) {\tiny $\partial_2 f$};

  \path (3.8,-2.2) edge[thin] (3.8,0);
  \node at (3.8,0.3) {\tiny $\partial_n f$};
  \node at (3.1,0.3) {\tiny $\cdots$};
  
  \node at (2.5,-5) {$\mathbb{D}(f)$};
  \node[text=gray] at (2.5,-5.5) {$5(s-2) + 10$ edges};


  \draw[fill=gray!20] (7,-3) -- ++(5,0) -- ++(-1,1) -- ++(-3,0) -- cycle;
  \node[leaf] at (7.3,-3.5) {\tiny $\;v\;$}
  edge[-*] (7.3,-2.8);
  \node[leaf] at (7.9,-3.5) {\tiny $x_1$}
  edge[-*] (7.9,-2.8);
  \node[leaf] at (8.5,-3.5) {\tiny $x_2$}
  edge[-*] (8.5,-2.8);

  \path (8.1,-2.2) edge (8.1,-1.8);
  \node at (8.1,-1.6) {\tiny $\partial_vf'$};

  \path (8.8,-2.2) edge (8.8,-1.8);
  \node at (8.8,-1.6) {\tiny $\partial_{1}f'$};

  \path (9.5,-2.2) edge (9.5,-1.8);
  \node at (9.5,-1.6) {\tiny $\partial_{2}f'$};

  \path (10.8,-2.2) edge[thin] ++(0,0.4);
  \node at (10.8,-1.6) {\tiny $\partial_{n}f'$};

  \node at (10.1,-1.6) {\tiny $\cdots$};
  \node at (9.5,-5) {$\mathbb{D}(f')$};
  \node[text=gray] at (9.5,-5.5) {$5(s-2)$ edges};

  \draw[-stealth,ultra thick] (5,4.3) -- (7,4.3);
  \draw[-stealth,ultra thick] (7,-2) -- (5,-2);
  \draw[-stealth,ultra thick] (9.5,2) -- (9.5,-1);
\end{tikzpicture}
\end{center}
\caption{Baur-Strassen pictorially for $v = x_1 \times x_2$}
\label{fig:baur-strassen}
\end{figure}


\section{Lower bounds for formulas}\label{sec:Kalorkoti}

This section would be devoted to the proof of Kalorkoti's lower bound \cite{k85} for formulas computing $\Det_n$, $\Perm_n$.

\begin{theorem}[\cite{k85}]\label{thm:kalorkoti}
  Any arithmetic formula computing $\Perm_n$ (or $\Det_n$) requires $\Omega(n^3)$ size.
\end{theorem}

The exploitable weakness in this setting is again to use the fact that the polynomials computed at intermediate gates share many polynomial dependencies.

\begin{definition}[Algebraic independence]
  A set of polynomials $\inbrace{f_1,\dots, f_m}$ is said to be \emph{algebraically independent} if there is no non-trivial polynomial $H(z_1,\dots, z_m)$ such that $H(f_1,\dots, f_m)=0$.

  The size of the largest algebraically independent subset of $\vecf=\inbrace{f_1,\dots, f_m}$ is called the \emph{transcendence degree} (denoted by $\mathrm{trdeg}(f)$).
\end{definition}
The proof of Kalorkoti's theorem proceeds by defining a \emph{complexity measure} using the above notion of algebraic independence.
\\


{\bf The Measure:} 
For any subset of variables $Y\subseteq X$, we can write a polynomial $f \in \F[X]$ of the form $f = \sum_{i=1}^s f_i \cdot m_i$ where $m_i$'s are distinct monomials in the variables in $Y$, and $f_i \in F[X \setminus Y]$.
We shall denote by $\mathrm{td}_Y(f)$ the transcendence degree of $\inbrace{f_1,\dots, f_s}$


Fix a partition of variables $X = X_1 \sqcup \dots \sqcup X_r$.
For any polynomial $f\in \F[X]$, define the map $\CM{Kal}:\F[X]\rightarrow \Z_{\geq 0}$ as
$$
\CM{Kal}(f) \quad=\quad \sum_{i=1}^r \mathrm{td}_{X_i}(f).
$$

The lower bound proceeds in two natural steps:
\begin{enumerate}
\item Show that $\CM{Kal}(f)$ is \emph{small} whenever $f$ is computable by a \emph{small} formula. 
\item Show that $\CM{Kal}(\Det_n)$ is \emph{large}. 
\end{enumerate}

\subsection{Upper bounding $\CM{Kal}$ for a formula}

\begin{lemma}\label{lem:kal-upperbound}
  Let $f$ be computed by a fan-in two formula $\Phi$ of size $s$.
Then for any partition of variables $X = X_1\sqcup \dots \sqcup X_r$, we have $\CM{Kal}(f) = O(s)$.
\end{lemma}
\begin{proof}
  For any node $v\in \Phi$, let $\textsc{Leaf}(v)$ denote the leaves of the subtree rooted at $v$ and let $\textsc{Leaf}_{X_i}(v)$ denote the leaves of the subtree rooted at $v$ that are in the part $X_i$.
Since the underlying graph of $\Phi$ is a tree, it follows that the size of $\Phi$ is bounded by a twice the number of leaves.
For each part $X_i$, we shall show that $\mathrm{td}_{X_i}(f) = O(\abs{\textsc{Leaf}_{X_i}(\Phi)})$, which would prove the required bound.
\\

Fix an arbitrary part $Y = X_i$.
Since the goal is really to just get a bound on $\textsc{Leaf}_{Y}(\Phi)$, we shall modify the formula $\Phi$ by introducing new leaf variables that compute polynomial in $\F[X \setminus Y]$.
This does not affect the size of $\textsc{Leaf}_{Y}(\Phi)$.
This in some sense is allowing $\Phi$ to ``freely'' compute any polynomial on variables outside $Y$, as we are only interested in how many times the $Y$ variables are used in the computation of $\Phi$.

We modify $\Phi$ by introducing a new type of gate $\boxtimes$ that takes a node $g$ and two leaves $\ell_1$ and $\ell_2$ and computes $\boxtimes(g,\ell_1,\ell_0) = (\ell_1 \cdot g) + \ell_0$.
We shall always ensure that when we introduce new leaf nodes, they only hold polynomials $\F[X \setminus Y]$.

Define the following three sets of nodes:
  \begin{eqnarray*}
    V_0 & = & \setdef{v\in \Phi}{\abs{\textsc{Leaf}_{Y}(v)} = 0 \quad\text{and}\quad \abs{\textsc{Leaf}_{Y}(\textsc{Parent}(v))} \geq 2}\\
    V_1 & = & \setdef{v\in \Phi}{\abs{\textsc{Leaf}_{Y}(v)} = 1 \quad\text{and}\quad \abs{\textsc{Leaf}_{Y}(\textsc{Parent}(v))} \geq 2}\\
    V_2 & = & \setdef{v\in \Phi}{\abs{\textsc{Leaf}_{Y}(v)} \geq 2}.
  \end{eqnarray*}


  Each node $v\in V_0$ computes a polynomial in $f_v \in \F[X\setminus Y]$, and we shall replace the subtree at $v$ by a leaf computing the polynomial $f_v$.

  Similarly, any node $v\in V_1$ computes a polynomial of the form $f_1 \cdot y_v + f_0$ for some $y_v\in Y$
  and $f_0,f_1 \in \F[X\setminus Y]$. 
  We shall create two new leaf nodes $\ell_0,\ell_1$ computing $f_0$ and $f_1$ respectively, and replace the gate $v$ by a $\boxtimes$ gate with inputs $y_v,\ell_1,\ell_0$ so that it computes $\boxtimes(y_v,\ell_1,\ell_0) = (f_1 \cdot y_v) + f_0$

  Hence, the formula $\Phi$ now reduces to a smaller formula $\Phi_Y$ with
  leaves  being the nodes in $V_0$ and $V_1$ (and nodes in $V_2$ are
  unaffected). 
  Furthermore, all new leaves that are added compute polynomials in $\F[X \setminus Y]$ and hence $\textsc{Leaf}_Y$ is unchanged. 
We would like to show that the size of the reduced
  formula, which is at most twice the number of its leaves, is
  $O(\abs{\textsc{Leaf}_Y(\Phi)})$.

  \begin{observation}\label{obs:v1-bound}$\abs{V_1} \leq \abs{\textsc{Leaf}_Y(\Phi)}$.    
  \end{observation}
  \begin{myproof}{Obs}
    Each node in $V_1$ has a distinct leaf labelled with a variable in $Y$.
Hence, $\abs{V_1}$ is bounded by the number of leaves labelled with a variable in $Y$.
  \end{myproof}
  
  This shows that the $V_1$ leaves are not too many.
Unfortunately, we cannot immediately bound the number of $V_0$ leaves, since we could have a long chain of $V_2$ nodes each with one sibling being a $V_0$ leaf.
The following observation would show how we can eliminate such long chains.

  \begin{observation}\label{obs:same-leaf-collapse}
    Let $u$ be an arbitrary node, and $v$ be another node in the subtree rooted at $u$ with $\textsc{Leaf}_Y(u) = \textsc{Leaf}_Y(v)$.
Then the polynomial $g_u$ computed at $u$ and the polynomial $g_v$ computed at $v$ are related as $g_u = f_1 g_v + f_0$ for some $f_1,f_0 \in \F[X\setminus Y]$.
  \end{observation}
  \begin{myproof}{Obs}
    If $\textsc{Leaf}_Y(u) =\textsc{Leaf}_Y(v)$, then every node on
    the path from $u$ to $v$ must have a $V_0$ leaf as the other child. 
The
    observation follows as all these nodes are $+$ or $\times$ gates.
  \end{myproof}

  Using the above observation, we shall remove the need for $V_0$
  nodes completely by adding new $\F[X\setminus Y]$ leaves and $\boxtimes$ gates. 
  Formally, if a $V_0$ node computing $f$ was connected to a $+$ node that was computing $f + g$, then we can replace the $V_0$ node by a new leaf $\ell_f$ computing $f$ and replace that $+$ node with $\boxtimes(g,1,\ell_f) = (g \cdot 1) + f$. Similarly, if a $V_0$ node computing $f$ was incident on a $\times$ node that was computing $f \cdot g$, then we can replace the $V_0$ node by a new leaf $\ell_f$ computing $f$ and replace the $\times$ gate with a $\boxtimes(g,\ell_f,0) = g \cdot f + 0$. 

Furthermore, using \autoref{obs:same-leaf-collapse}, we can now contract any two nodes $u$ and $v$ with $\textsc{Leaf}_Y(u) = \textsc{Leaf}_Y(v)$, we can replace the entire chain from $u$ to $v$ by an appropriate $\boxtimes$ node. 
This ensures that no $\boxtimes$ node is connected to another $\boxtimes$ node. 
Overall, $\Phi$ has been transformed in the following ways:
\begin{enumerate}
\itemsep 0pt
\item all $V_0$ nodes are replaced by leaves,
\item all $V_1$ are replaced by $\boxtimes$ nodes with three leaves incident on it,
\item no $\boxtimes$ node is incident on another $\boxtimes$ node, and hence its parent is a $V_2$ node,
\item distinct vertices $u,v\in V_2$ satisfy $\textsc{Leaf}_Y(u) \neq \textsc{Leaf}_Y(v)$, 
\item leaves in $Y$ are untouched,
\item all new leaves compute polynomials only in $\F[X \setminus Y]$.
\end{enumerate}
(2) implies that the number of $|V_1|$ nodes in $\hat{\Phi}_Y$ is at most $\abs{\textsc{Leaf}_Y(\Phi)}$.
Also (4) implies the number of $V_2$ nodes is at most $\abs{\textsc{Leaf}_Y(\Phi)} - 1$.
Therefore, the size of the augmented formula $\hat{\Phi}_Y$ is at most $2\abs{\textsc{Leaf}_Y(\Phi)}$.
\\

Suppose $\Phi$ computes a polynomial $f$, which can be written as $f = \sum_{i=1}^t f_i\cdot m_i$ with $f_i \in \F[X\setminus Y]$ and $m_i$'s being distinct monomials in $Y$.
Since $\hat{\Phi}_Y$ also computes $f$, each $f_i$ is a polynomial combination of the leaves of $\hat{\Phi}_Y$ that compute polynomials in $\F[X \setminus Y]$.
Since $\hat{\Phi}_Y$ consists of at most $2\abs{\textsc{Leaf}_Y(\Phi)}$ augmented nodes, we have that $\mathrm{td}_Y(f) \leq 4\abs{\textsc{Leaf}_Y(\Phi)}$.
Therefore,
$$
\mathrm{td}_Y(f) \quad=\quad \mathrm{trdeg}\setdef{f_i}{i\in [t]} \quad\leq\quad 4\abs{\textsc{Leaf}_Y(\Phi)}
$$
Hence, 
$$
\CM{Kal}(\Phi) = \sum_{i=1}^r \mathrm{td}_{X_i}(f_i) \leq 4\inparen{\sum_{i=1}^r \abs{\textsc{Leaf}_{X_i}}} = O(s).\qedhere
$$
\end{proof}

\subsection{Lower bounding $\CM{Kal}(\Det_n)$}


\begin{lemma}\label{lem:kal-lowerbound}
  Let $X = X_1 \sqcup \dots \sqcup X_n$ be the partition as defined by
  $X_t = \setdef{x_{ij}}{i-j\equiv t\bmod{n}}$. 
Then,
  $\CM{Kal}(\Det_n) = \Omega(n^3)$.
\end{lemma}
\begin{proof}
  By symmetry, it is easy to see that $\text{td}_{X_i}(\Det_n)$ is
  the same for all $i$. 
Hence, it suffices to show that
  $\text{td}_{Y}(\Det_n) = \Omega(n^2)$ for $Y = X_n = \inbrace{x_{11},\dots, x_{nn}}$. 

  To see this, observe that the determinant consists of the monomials
  $\inparen{\frac{x_{11}\dots x_{nn}}{x_{ii}x_{jj}}}\cdot
  x_{ij}x_{ji}$ for every $i\neq j$. 
Hence, $\text{td}_{Y}(\Det_n)
  \geq \mathrm{trdeg}\setdef{x_{ij}x_{ji}}{i\neq j} =
  \Omega(n^2)$. 
Therefore, $\CM{Kal}(\Det_n) =
  \Omega(n^3)$.
\end{proof}

The proof of \autoref{thm:kalorkoti} follows from
\autoref{lem:kal-upperbound} and \autoref{lem:kal-lowerbound}.



%%% Local Variables: 
%%% mode: latex
%%% TeX-master: "main"
%%% End: 
