\chapter{Some estimates for binomial coefficients}\label{chap:binom-estimates}

Throughout this article, we would be seeing several binomial coefficients.  
The following estimates would allow us to get a better handle on the growth of such terms. 

We shall use $\log$ to refer to $\log_2$ and $\ln$ to refer to the natural logarithm. 


\begin{definition}[Entropy function]\label{def:entropy}
The binary entropy function $H_2:[0,1]\rightarrow [0,1]$ is defined as
\[
H_2(p) \quad = \quad - p \log_2(p) \spaced{-} (1-p) \log_2(1-p)
\]
The entropy function with respect to the natural logarithm is refer to as $H$ and
\[
H(p) \quad = \quad - p \ln(p) \spaced{-} (1-p) \ln(1-p)
\]
\end{definition}

\begin{proposition}\label{prop:entropy-estimate}
For any $0< p  < 1$, we have $p\ln\frac{1}{p} \leq H(p) \leq p\ln\frac{1}{p} + p$. 
\end{proposition}

\begin{proposition}[Stirling's Approximation]\label{prop:sterling}
\[
\lim_{n\rightarrow \infty}\frac{n!}{\inparen{\frac{n}{e}}^n \sqrt{2\pi n}}  \quad=\quad 1
\]
\end{proposition}

\begin{proposition}
For any constants $\alpha, \beta$, 
\[
\log\binom{\alpha n}{\beta n} \quad = \quad H_2\inparen{\frac{\beta}{\alpha}} \cdot \alpha n - O(\log n)
\]
In particular, if $\beta = \alpha/2$, then $\binom{\alpha n}{\beta{n}} = 2^{\alpha n} / \poly(n)$. 
\end{proposition}

For the more recent lower bounds, we would encounter several delicate ratios of binomial coefficients. 
The following lemma would help us simplify several such expressions and get a better handle on the growth. 

\begin{lemma}{\cite[Lemma 6]{gkks13}}\label{lem:factorial-ratio} For any $a,b = O(\sqrt{n})$, then
\[
\frac{(n+a)!}{(n-b)!}\quad=\quad n^{a + b} \cdot \poly(n)
\]
\end{lemma}

We shall be using the above lemma very often in the lower bounds. 
One particular instantiation that shall also appear frequently shall be the following lemma.  

\begin{restatable}{lemma}{binomapprox}\label{lem:binom-approx}
Let $n$ and $\ell$ be parameters such that $\ell = \frac{n}{2}(1 - \epsilon)$ for some $\epsilon = o(1)$. 
For any $a, b$ such that $a,b = O(\sqrt{n})$, 
\[
\binom{n - a}{\ell - b} \quad = \quad \binom{n}{\ell} \cdot 2^{-a} \cdot (1+\epsilon)^{a-2b} \cdot \exp(O(b\cdot \epsilon^2))
\]
\end{restatable}
\begin{proof}
The proof of the above lemma would repeated use \autoref{lem:factorial-ratio}. 
\begin{eqnarray*}
\frac{\binom{n-a}{\ell -b}}{\binom{n}{\ell}} & = & \frac{(n-a)!}{n!} \cdot \frac{\ell!}{(\ell -b)!}\cdot \frac{(n-\ell)!}{(n-\ell-a+b)!}\\
& \stackrel{\poly}{\approx}& \frac{1}{n^a} \cdot \ell^b \cdot \frac{(n-\ell)^a}{(n-\ell)^b}\\
& = & \frac{\inparen{\frac{n}{2}}^a(1 +\epsilon)^a}{n^a} \cdot \frac{(1-\epsilon)^{b}}{(1+\epsilon)^b}\\
& = & 2^{-a} \cdot (1+\epsilon)^{a - 2b} \cdot \exp(O(b\cdot \epsilon^2))
\end{eqnarray*}
\end{proof}


%%% Local Variables: 
%%% mode: latex
%%% TeX-master: "main"
%%% End: 
