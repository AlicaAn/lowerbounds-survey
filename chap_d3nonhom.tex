\chapter{The power of non-homogeneous depth three circuits}

A $\Sigma\Pi\Sigma$ circuit computes a polynomial of the form
\[
f\quad=\quad \sum_{i=1}^s \ell_{i1}\dots \ell_{iD}.
\]
If the circuit is non-homogeneous, the degree of the circuit $D$ could potentially be much larger than $\deg(f)$. \\

The class of depth three arithmetic circuits can compute polynomials
in non-trivial ways.
To illustrate a couple of examples, there is a homogeneous $\SPS$
circuit for $\Perm_n$ of size $2^{O(n)}$ called Ryser's Formula
\cite{rys63}
\begin{equation}\label{eqn:ryser}
\Perm_n \spaced{=} \sum_{S \subseteq[n]} (-1)^{n - |S|} \prod_{i=1}^n \inparen{\sum_{j\in S} x_{ij}}
\end{equation}
On the other hand, no $\SPS$ circuit for the $\Det_n$ significantly
better than writing it as a sum of $n!$ monomials was known (until
\cite{gkks13b}). 
Further, the elementary symmetric polynomials $\ESym_k(x_1,\dots,
x_n)$ of degree $k$ defined as
\[
\ESym_k(\vecx) \spaced{=} \sum_{\substack{S\subset \vecx\\|S| = k}} \prod_{x_i\in S}x_i
\]
can be computed by a non-homogeneous depth $3$ circuit of size
$O(n^2)$ over any characteristic zero field.
In stark contrast, \cite{nw1997} showed that any homogeneous depth $3$
circuit computing $\ESym_k$ requires size
$n^{\Omega(k)}$. \cite{nw1997} also showed a $2^{\Omega(n)}$ lower
bound for homogeneous depth $3$ circuits computing $\Perm_n$ or
$\Det_n$. 

Also, the results of \cite{gr00,grigoriev98} showed a $2^{\Omega(n)}$
lower bound for $\SPS$ circuits \emph{over finite fields} that compute
$\Det_n$ or $\Perm_n$.
All these results seemed to suggest that there perhaps is an
$2^{\Omega(n)}$ lower bound for $\SPS$ circuits computing $\Det_n$
over characteristic zero fields as well.
If it was true over finite fields, and for homogeneous $\SPS$
circuits, how much power can characteristic zero fields and
non-homogeneity add? 
As it turns out, quite a lot!

\begin{theorem}[\cite{gkks13b}] \label{thm:chasm-at-3} Let $f$ be an
  $n$-variate degree $d$ polynomial computed by an arithmetic circuit
  of size $s$ over any characteristic zero field.
  Then there is a $\SPS$ circuit of size $s' \leq s^{O(\sqrt{d})}$
  that computes $f$. 
\end{theorem}
\begin{corollary}[\cite{gkks13b}]\label{cor:det-sps}
There is a $\SPS$ circuit over $\Q$, the field of rational numbers, of size $n^{O(\sqrt{n})}$. 
\end{corollary}


The proof is quite short and comprises of two steps using known reductions, and going through a bizarre intermediate model of \emph{depth $5$ powering circuits}.
We present a longer route towards this result that perhaps sheds more light on the reduction.

\section{Handling non-homogeneous depth-$3$ circuits}\label{sec:non-hom-d3}

Non-homogeneous models are generally difficult to deal with in the
context of lower bounds.
A natural question to ask if any such non-homogeneous circuit can be
converted to a suitable homogeneous model which may then be attacked.
This was first studied by Shpilka and Wigderson~\cite{sw2001}. \\

Let $f$ be a homogeneous degree $d$ polynomial computed by a possibly
non-homogeneous depth $3$ circuit $C$ of the form
\[
f\quad=\quad \sum_{i=1}^s \ell_{i1}\dots \ell_{iD}
\]
As a first step, let us extract the degree $d$ homogeneous component
of each summand on the RHS.
Since $f$ is a homogeneous degree $d$ polynomial, $f$ has to be sum of
the degree $d$ homogeneous components of each summand on the RHS.
Consider a single term of the form
\[
T \spaced{=} (\ell_1 + \alpha_1)\cdots (\ell_D + \alpha_D)
\]
where each $\ell_i$ is a homogeneous linear polynomial, and $\alpha$
are elements from the field.
Assuming that the first $r$ of the $\alpha_i$'s are zero, we can write
$T$ in the form (with some reuse of symbols)
\begin{eqnarray*}
T & = &  \alpha \cdot \ell_1\dots \ell_r \cdot (\ell_{r+1} + 1)\dots (\ell_D+1)\\
\implies \quad \mathrm{Hom}_d(T) & = & \ell_1 \dots \ell_r \cdot \ESym_{d-r}(\ell_{r+1}, \dots, \ell_D)
\end{eqnarray*}
where $\ESym_{k}(\vecx)$, the elementary symmetric polynomial of
degree $k$ defined as
\[
\ESym_k(\vecx) \spaced{=} \sum_{\substack{S\subset \vecx\\|S| = k}} \prod_{x_i\in S}x_i
\]
Hence, if we can show that $\ESym_{d-r}(\vecx)$ has a not-too-large
homogeneous depth $4$ circuit, then we can immediately infer that $f$
can be computed by a not-too-large homogeneous depth $5$ circuit.
The following identities, attributed to Newton (cf. \cite{kalman00}),
is exactly what we need.
Define the \emph{power symmetric polynomials}, denoted by
$\mathrm{Pow}_k(\vecx)$ as
\[
\mathrm{Pow}_k(\vecx) \quad = \quad \sum_{x_i \in \vecx} x_i^k
\]

\begin{lemma}[Newton Identities]\label{lem:newton-identities}
  Let $\ESym_k(x_1,\ldots,x_m)$ and $\PSym_k(x_1,\ldots,x_m)$ denote
  the \emph{elementary symmetric} and \emph{power symmetric} polynomials of degree
  $k$ respectively, as defined above. 
Then,
  $$
  \ESym_k\spaced{=}\frac{1}{k!} \cdot 
  \begin{vmatrix}\PSym_1 & 1 & 0 & \cdots & 0 & 0\\ 
    \PSym_2 & \PSym_1 & 2 & \cdots & 0 & 0 \\ 
    \vdots& \vdots & \ddots & \ddots &  & \vdots \\
    \vdots& \vdots &  & \ddots & \ddots & \vdots \\
    \PSym_{k-1} & \PSym_{k-2} & \PSym_{k-3}& \cdots & \PSym_1 & k-1 \\ 
    \PSym_k & \PSym_{k-1} & \PSym_{k-2} & \cdots & \PSym_2 & \PSym_1 
  \end{vmatrix}.
  $$
\end{lemma}

Expanding the determinant on the RHS, we obtain a homogeneous
expression
\begin{equation}\label{eqn:esym}
\ESym_k(\vecx) \quad = \quad \sum_{\veca\;:\; \sum_i ia_i = k} \alpha_\veca \cdot (\PSym_1)^{a_1}\dots (\PSym_k)^{a_k}
\end{equation}
The number of summands bounded by the number of non-negative solutions
to $\sum i a_i = k$ , which is precisely the number of partitions of
$k$.
By the estimates of \cite{hr18}, we know that the number of partitions
of $k$ is bounded by $2^{\Theta(\sqrt{k})}$.
Thus, (\ref{eqn:esym}) yields a homogeneous depth $4$ circuit for
$\ESym_k(x_1,\dots, x_m)$ of size $2^{\Theta(\sqrt{k})} \cdot m$.
In fact, the circuit is a homogeneous $\SPSE$ circuit, i.e. a $\SPSP$
circuit where the bottom layer of multiplication in fact just raises a
single variable to a higher power. 

Let us summarizing this as a lemma. 

\begin{lemma}[\cite{sw2001}]\label{lem:d3-d5} For every $d \leq n$, the elementary
  symmetric polynomial $\ESym_d$ can be computed by a homogeneous $\SPSE$ circuit of size $2^{O(\sqrt{d})} \cdot \poly(n)$ over any field $\F$ of characteristic zero.

  Thus, if $f$ is a homogeneous degree $d$ polynomial computed by a (possibly non-homogeneous) depth-$3$ circuit $C$ of size $s$ over a field $\F$ of characteristic zero, then $f$ can be equivalently computed by a $\Sigma\Pi\SES$ of size at most $2^{O(\sqrt{d})} \cdot \poly(s)$.
\end{lemma}

\section{Depth reduction to depth three circuits}\label{sec:depth-3-red}

In this section we shall see the proof of the depth reduction of
\cite{gkks13b}.
As mentioned earlier, the proof is quite short but we shall give a
slightly lengthier exposition that is close to how the result was
discovered.
This route via an attempt to prove better lower bounds for depth-$4$
circuits might be more insightful than the actual proof itself. 

\subsubsection{Towards proving better lower bounds for depth $4$ circuits}

From the depth reduction to depth-$4$ (\autoref{thm:av}), it suffices to
prove a better lower bound for explicit polynomials computed as
\begin{equation}\label{eqn:d4-LB}
f\spaced{=} \sum_{i=1}^s Q_{i1}\dots Q_{ir} \quad\text{where}\quad \deg(Q_{ij})\leq \sqrt{d}\;,\; r\leq O(\sqrt{d})
\end{equation}
A noble goal is to show a lower bound of $s = n^{\omega(\sqrt{d})}$.
Perhaps a simpler question to ask is to prove a lower bound for
expressions of the form
\begin{equation}\label{eqn:d4pow-LB}
f\spaced{=} \sum_{i=1}^s Q_{i}^{\sqrt{d}} \quad\text{where }\deg(Q_i)\leq \sqrt{d}
\end{equation}
Fortunately, if the goal is to prove lower bounds of
$n^{\omega(\sqrt{d})}$, then without loss of generality we can focus
on this equation instead.

\begin{lemma}\label{lem:fischer}
  Over any characteristic zero field, given an expression of the form
  \[
  f\spaced{=} \sum_{i=1}^s Q_{i1}\dots Q_{ir} \quad\text{where}\quad \deg(Q_{ij})\leq \sqrt{d}\;,\; r\leq O(\sqrt{d})
  \]
  there is an equivalent equation
  \[
  f\spaced{=} \sum_{i=1}^{s'} Q_{i}^r\quad\text{where}\quad \deg(Q_{i})\leq \sqrt{d}
  \]
  with $s' \leq s \cdot 2^{O(\sqrt{r})}$. 
\end{lemma}
\begin{proof}
  Consider Ryser's formula (\ref{eqn:ryser}) applied for to the
  $r\times r$ matrix where each row is $[y_1,\dots, y_r]$. 
  \[
  \Perm \insquare{\begin{array}{ccc} y_1 & \dots & y_r\\ \vdots & \ddots & \vdots \\ y_1 & \dots & y_r\end{array}} \quad = \quad r! \cdot y_1\dots y_r \quad = \quad \sum_{S\subseteq [r]} \inparen{\sum_{j\in S} y_j}^n
  \]
  The lemma follows by applying this identity on each term
  $Q_{i1}\dots Q_{ir}$. 
\end{proof}

A very similar identity, to convert a product into sums of powers of
linear polynomials, is often attributed to Fischer \cite{fischer}.
We shall refer to this as the Ryser-Fischer trick. 

\begin{lemma}[Ryser-Fischer Trick]\label{lem:ryser-fischer}
$$y_1 \dots y_r  = \frac{1}{r!} \cdot \sum_{S\subseteq [r]} \inparen{\sum_{j\in S} y_j}^n$$
\end{lemma}

Note that since we need to divide by $r!$, the above lemma fails over
low characteristic fields, in particular finite fields.
Thus, proving an $n^{\omega(\sqrt{d})}$ lower bound for expressions
such as (\ref{eqn:d4pow-LB}) implies an $n^{\omega(\sqrt{d})}$ lower
bound for expressions such as (\ref{eqn:d4-LB}).
We shall call expressions such as (\ref{eqn:d4pow-LB}) as
$\Sigma\mywedge\Sigma\Pi^{[\sqrt{d}]}$ circuits. 

Just as we converted the top $\Pi$ layer into powering layers using
the Ryser-Fischer identity, the same can be done to the lower layer of
$\Pi$ gates as well.

\begin{corollary}\label{cor:pow-genckt}
  If a homogeneous $n$-variate degree $d$ polynomial $f$ can be
  computed by a $\SPSPfanin{O(\sqrt{d})}{\sqrt{d}}$ of size $s =
  n^{O(\sqrt{d})}$, then $f$ can also be computed by an
  $\Sigma\mathord{\wedge^{[O(\sqrt{d})]}}\Sigma\mathord{\wedge^{[\sqrt{d}]}}\Sigma$
  circuit of size $s' = s \cdot 2^{O(\sqrt{d})}$. 

  Conversely, if $f$ requires
  $\Sigma\mathord{\wedge^{[O(\sqrt{d})]}}\Sigma\mathord{\wedge^{[\sqrt{d}]}}\Sigma$
  circuits of size $s' = n^{\omega(\sqrt{d})}$ to compute it, then $f$
  cannot be computed by polynomial sized arithmetic circuits. 
\end{corollary}

We shall take a small detour to apply this to the conversion of non-homogeneous depth $3$
circuits  to homogeneous shallow circuits.

\subsubsection{Revisiting non-homogeneous depth $3$ circuits}

From \autoref{sec:non-hom-d3}, we know that any non-homogeneous $\SPS$
circuit can be converted to a homogeneous $\Sigma\Pi\SES$ circuit, and
this was essentially by writing elementary symmetric polynomial
$\ESym_d$ has a homogeneous $\SPSE$ circuit of size $2^{O(\sqrt{d})}
\cdot \poly(n)$:
\[
\ESym_d(\vecx) \quad = \quad \sum_{\veca\;:\; \sum_i ia_i = d} \alpha_\veca \cdot (\PSym_1)^{a_1}\dots (\PSym_d)^{a_d}
\]

To convert this $\Sigma\Pi\Sigma\mywedge$ circuit to a
$\Sigma\mywedge\Sigma\mywedge$ circuit, we could use
Ryser-Fischer's identity again.
At first sight, it appears as though this would yield a blow up of
$2^d$ as some of the product gates could have fan-in $d$.
However, notice that the sum is over $a_i$'s satisfying $\sum i\cdot
a_i = d$.
Hence, there can be at most $O(\sqrt{d})$ of the $a_i$'s that are
non-zero.
By looking at Ryser-Fischer's identity applied on $y_1^{a_1}\dots
y_{d}^{a_d}$ more carefully, we see that it uses at most $(1+a_1)\dots
(1+a_d) \leq d^{O(\sqrt{d})}$ distinct linear polynomials instead of the
\naive bound of $2^{d}$.
This fact of expressing any degree $d$ monomial over $m$ variables as
a $\Sigma\mywedge\Sigma$ circuit of size $d^{O(m)}$ was also observed
by Ellison \cite{ellison}. 

Thus, if $f$ admits a poly-sized depth three circuit (possibly non-homogeneous), then $f$ also
admits a homogeneous $\Sigma\mywedge\Sigma\mywedge\Sigma$ circuit of
size $d^{O(\sqrt{d})} \cdot \poly(n)$.
The following lemma summarizes this discussion. 

\begin{lemma}\label{lem:pow-depth3}
  Let $f$ be an $n$-variate degree $d$ polynomial that is computable
  by depth three circuit of size $s$ over $\Q$.
  Then, $f$ is equivalently computable by a homogeneous
  $\Sigma\mywedge\Sigma\mywedge\Sigma$ circuit of size
  $d^{O(\sqrt{d})}\cdot \poly(s)$. 

  Conversely, if $f$ requires $\Sigma\mywedge\Sigma\mywedge\Sigma$
  circuits of size $n^{\omega(\sqrt{d})}$ over $\Q$ to compute it,
  then $f$ requires depth three circuits of size
  $n^{\omega(\sqrt{d})}$. 
\end{lemma}

In fact, this bound can be improved further and we shall address this
shortly. 

\subsubsection{Completing the picture}

\begin{figure}
\begin{center}
\begin{tikzpicture}
\node (SESES) at (0,0) {$\begin{array}{c}n^{\omega(\sqrt{d})} \text{ LB} \\\text{ for } \SESES \text{ circuits}\end{array}$};
\node (genCkt) at (-2.5,-2) {$\begin{array}{c} n^{\omega(1)} \text{ LB} \\ \text{ for general circuits} \end{array}$}
edge[stealth-, very thick] (SESES);
\node (SEPS) at (2.5,-2) {$\begin{array}{c} n^{\omega(\sqrt{d})} \text{ LB} \\ \text{ for $\SPS$ circuits} \end{array}$}
edge[stealth-, very thick] (SESES)
edge[->, draw=gray, dashed, thick] (genCkt);
\node[text=gray] at (0,-1.7) {??};
\end{tikzpicture}
\end{center}
\caption{Power of $\SESES$ ckts.}
\label{fig:SESES}
\end{figure}


We now have an interesting situation (\autoref{fig:SESES}).
On one hand, \autoref{cor:pow-genckt} states that a lower bound of
$n^{\omega(\sqrt{d})}$ for $\SESES$ circuits would yield a
super-polynomial lower bound for general arithmetic circuits.
On the other, \autoref{lem:pow-depth3} states that an
$n^{\omega(\sqrt{d})}$ lower bound for $\SESES$ circuits would yield a
lower bound of $n^{\omega(\sqrt{d})}$ for depth three circuits. 

Could this just be a coincidence? 
Or, is it the case that any
poly-sized arithmetic circuit can be equivalently expressed as a depth
three circuit of size $n^{O(\sqrt{d})}$ over $\Q$? 
As it turns out,
there is indeed a depth reduction to convert any arithmetic circuit to
a not-too-large depth three circuit over $\Q$. \\

To complete the picture, it suffices to show that a
$\mywedge\Sigma\mywedge$ circuit can be expressed as a
$\Sigma\Pi\Sigma$ circuit.
This would automatically imply a reduction from $\SESES$ circuits to
$\Sigma\Pi\Sigma$ circuits.
The last step of the puzzle is the \emph{duality trick} of
\cite{sax08}. 
A similar version of this trick also appeared in the lower bound of Shpilka and Wigderson~\cite{sw2001} but this statement is from the work of Saxena~\cite{sax08}.

\begin{lemma}[The Duality Trick \cite{sax08}]\label{lem:duality} There exists univariate polynomials $f_{ij}$'s of degree at most $b$ such that
$$
\inparen{z_1 + \dots + z_s}^b \quad = \quad \sum_{i=1}^{sb+1}
f_{i1}(z_1)\dots f_{is}(z_s).
$$
\end{lemma}

It is worth noting that the degree of each term on the RHS is $sb$,
whereas the LHS just has degree $b$.
This is the place where non-homogeneity is introduced.
Applying the above lemma for a $\mywedge\Sigma\mywedge$ circuit such
as $(y_1^a + \dots + y_s^a)^b$ gives

\begin{eqnarray*}
  (y_1^a + \dots + y_s^a)^b & = & \sum_{i=1}^{sb+1} \prod_{j=1}^s f_{ij}(y_j^a)\\
  & = & \sum_{i=1}^{sb+1} \prod_{j=1}^s \tilde{f}_{ij}(y_j)
\end{eqnarray*}
where $\tilde{f}_{ij}(y) = f_{ij}(y^a)$.
Since each $\tilde{f}_{ij}(y)$ is a univariate polynomial, it can be
factorized completely over the $\C$, the field of complex numbers.
Hence, if $f_{ij}(y) = \prod_k (y - \zeta_{ijk})$, then we get
\begin{eqnarray*}
  (y_1^a + \dots + y_s^a)^b & = & \sum_{i=1}^{sb+1} \prod_{j=1}^s \tilde{f}_{ij}(y_j)\\
  &= & \sum_{i=1}^{sb+1} \prod_{j=1}^s \prod_{k=1}^b (y_j - \zeta_{ijk})\\
\end{eqnarray*}
which is a depth three circuit! 
Thus, $(y_1^a + \dots + y_s^a)$ can be
expressed as a depth three circuit of size $\poly(s,a,b)$ over $\C$.
With a little more effort, one can construct a depth three circuit
over $\Q$ as well.
Summarizing this is a lemma, we have the following. 

\begin{lemma}
  Any $n$-variate degree $d$ polynomial $f$ computed by a homogeneous
  $\SESES$ of size $s$ over a characteristic zero field $\F$ can also
  be computed by a depth three circuit of size $\poly(s,n,d)$ over
  $\F$. 
\end{lemma}

Combining with \autoref{cor:pow-genckt} and \autoref{thm:av}, we obtain the main result of \cite{gkks13b}. \\

\noindent {\bf \autoref{thm:chasm-at-3} (restated). }{\em Let $f$ be
  an $n$-variate degree $d$ polynomial computed by an arithmetic
  circuit of size $s$ over any characteristic zero field.
  Then there is a $\SPS$ circuit of size $s' \leq s^{O(\sqrt{d})}$
  that computes $f$. 
}\\

{\bf Remark. } Note that if we were to start with a degree $d$
polynomial $f$ and apply the above depth reduction, all the linear
polynomials that we obtain at bottom are essentially from the
application of Ryser-Fischer's identity on the bottom $\Pi$ layer of
fanin $\sqrt{d}$ of the $\SPSPfanin{O(\sqrt{d})}{\sqrt{d}}$ circuit.
Hence, the each of the linear polynomials that appear in the final
$\SPS$ circuit depend on at most $\sqrt{d}$ variables.
In other words, the above Theorem yields a reduction to
$\SPS^{[\sqrt{d}]}$ circuits.

\section{Revisiting the depth-five powering circuit for $\ESym_d$ }

As mentioned earlier, the conversion of a non-homogeneous depth-$3$
circuit to a homogeneous depth-$5$ circuit proceeds via the
construction of Shpilka and Wigderson~\cite{sw2001} for the elementary
symmetric polynomial.
We present a slightly more careful study of the resulting depth-$5$ circuit of
this construction.

The first step in building the circuit for $\ESym_d$ was \eqref{eqn:esym}, to express it as
\[
\ESym_d(\vecx) \quad = \quad \sum_{\veca\;:\; \sum_i ia_i = d} \alpha_\veca \cdot (\PSym_1)^{a_1}\dots (\PSym_d)^{a_d}.
\]

In the next step, we used the Ryser-Fischer trick
(\autoref{lem:ryser-fischer}) to replace the top $\times$-gate by a
$\Sigma\mathord{\wedge}\Sigma$ circuit.
We then observed that a monomial $y_1^{a_1}\cdots y_d^{a_d}$ can be
expressed as a $\SES$ circuit of top fan-in at most $(1 + a_1)\cdots
(1+a_d)$.
Since we know that the above expression has $\sum i \cdot a_i = d$, at
most $O(\sqrt{d})$ of the $a_i$'s are non-zero and this gives a bound
of $d^{O(\sqrt{d})}$. 

However, one can very easily obtain a slightly better bound of
$2^{O(\sqrt{d})}$ instead of $d^{O(\sqrt{d})}$. 

\begin{proposition}\label{prop:sym-d5-am-gm}
If $a_1,\dots, a_d$ are non-negative integers with $\sum_{i=1}^d i \cdot a_i  = d$, then 
\[
(1+a_1)\cdots (1+a_d) \quad \leq \quad 2^{O(\sqrt{d})}
\]
\end{proposition}
\begin{proof}
  Without loss of generality, we may assume that the non-zero $a_i$'s
  are the first $r$ for some $r = O(\sqrt{d})$.
  Given $\sum i \cdot a_i = d$, we wish to infer a bound for
  $(1+a_1)\cdots (1+a_r)$ and a natural approach to do this is to use
  the AM-GM inequality somehow.
  Indeed,
  \begin{eqnarray*}
    (1 + a_1) + (2 + 2a_2) + \cdots + (r + r a_r) & = & \frac{r(r+1)}{2} + d \; \leq \;2d \\
    \implies    (1 + a_1) \cdot (2 + 2a_2) \cdots (r + r a_r) & \leq & \pfrac{2d}{r}^r \quad\text{(AM-GM inequality)}\\
    \implies    (1 + a_1) \cdot (1 + a_2) \cdots (1 + a_r) & \leq & \pfrac{2d}{r}^r \pfrac{1}{r!} \quad = \quad \pfrac{d}{r^2}^r \cdot \exp(r)
  \end{eqnarray*}
  which is $2^{O(\sqrt{d})}$ for all $r = O(\sqrt{d})$. 
\end{proof}

\begin{corollary}\label{cor:d5-pow-for-esym}
  For every $d \leq n$, the elementary symmetric polynomial $\ESym_d$
  can be computed by a homogeneous $\SESE$ circuit of size at most
  $2^{O(\sqrt{d})} \cdot \poly(n)$ over any field $\F$ of
  characteristic $0$.
\end{corollary}

Thus, specializing to non-homogeneous depth-$3$ circuits, we have the
following immediate corollary. 

\begin{corollary}\label{cor:d5-pow-for-nonhom-d3}
  Let $f$ be computed by a (possibly non-homogeneous) depth-$3$
  circuit $C$ of size $s$.
  Then, for every $d \leq \deg(f)$, the $d$-th homogeneous part of $f$
  can be computed by a $\SESES$ circuit of size $\exp(\sqrt{d}) \cdot
  poly(s)$.

  Further, any bound on the bottom fan-in of the circuit $C$
  translates to a similar bound on the bottom fan-in of the resulting
  $\SESES$ circuit. 
\end{corollary}

\autoref{cor:d5-pow-for-esym} in particular implies that the monomial
$x_1\cdots x_n$ has a $\SESE$ circuit of size $2^{O(\sqrt{n})}$.
This is pretty surprising as we know of a $2^{\Omega(n)}$ lower bound
for $\SES$ circuits computing the monomial.
Allowing higher powers at bottom layer might appear to be of no
assistance in computing a multilinear monomial but surprisingly it
does!\footnote{This observation is by Michael Forbes. }

\begin{openproblem}
Show a lower bound on the size of an $\SESE$ circuit computing the monomial $x_1\cdots x_n$. 
\end{openproblem}


%%% Local Variables: 
%%% mode: latex
%%% TeX-master: "fancymain"
%%% End: 
