\section{Weak lower bounds for general circuits and formulas}\label{sec:gen-ckt-formulas}

Despite several attempts by various researchers to prove lower bounds for arithmetic circuits or formulas, we only have very mild lower bounds for general circuits or formulas thus far. In this section, we shall look at the two  modest lower bounds for general circuits and formulas. 

\subsection{Lower bounds for general circuits}\label{sec:baur-strassen}

The only super-linear lower bound we currently know for general arithmetic circuits is the following  result of Baur and Strassen \cite{BS83}.

\begin{theorem}[\cite{BS83}]\label{thm:baur-strassen}
  Any fan-in $2$ circuit that computes the polynomial $f = x_1^{d+1} + \dots + x_n^{d+1}$ has size $\Omega(n\log d)$. 
\end{theorem}

\subsubsection{An exploitable weakness}

Each gate of the circuit $\Phi$ computes a local operation on the two children. To formalize this, define a new variable $y_g$ for every gate $g \in \Phi$. Further, for every gate $g$ define a quadratic equation $Q_g$ as
$$
Q_g = \begin{cases} y_g - (y_{g_1} + y_{g_2}) & \text{if $g = g_1 + g_2$}\\
  y_g - (y_{g_1}\cdot y_{g_2}) & \text{if $g = g_1 \cdot g_2$}.
\end{cases}
$$
Further if $y_o$  corresponds to the output gate, then the system of equations
$$\setdef{Q_g = 0}{g\in \Phi} \spaced{\union} \inbrace{y_{o} = 1}$$
completely characterize the computations of $\Phi$ that results in an output of $1$. 

The same can also be extended for \emph{multi-output} circuits that compute several polynomials simultaneously. In such cases, the set of equations
$$\setdef{Q_g = 0}{g\in \Phi} \spaced{\union} \setdef{y_{o_i} = 1}{i=1, \ldots, n}$$
completely characterize computations that result in an output of all ones. The following classical theorem allows us to bound the number of  common roots to a system of polynomial equations. 

\begin{theorem}[\Bezout's theorem]
  Let $g_1,\dots, g_r \in \F[X]$ such that $\deg(g_i) = d_i$ such that the number of common roots of $g_1=\dots=g_r = 0$ is finite. Then, the number of common roots (counted with multiplicities) is bounded by $\prod d_i$.
\end{theorem}

Thus in particular, if we have a circuit $\Phi$ of size $s$ that \emph{simultaneously} computes $\inbrace{x_1^d, \dots,x_n^d}$, then we have $d^n$ inputs that evaluate to all ones (where each $x_i$ must be  a $d$-th root of unity). Hence, \Bezout's theorem implies that
$$
2^s\spaced{\geq} d^n \spaced{\quad\implies\quad} s \spaced{=} \Omega(d\log n).
$$

Observe that $\inbrace{x_1^d,\dots, x_n^d}$ are all first-order derivatives of $f = x_1^{d+1}+\dots+x_n^{d+1}$ (with suitable scaling). A natural question here is the following --- if $f$ can be computed an arithmetic circuit of size $s$, what is the size required to compute all first-order partial derivatives of $f$ simultaneously? The \naive approach of computing each derivative separately results in a circuit of size $O(s\cdot n)$. Baur and Strassen \cite{BS83} show that we can save a factor of $n$.

\begin{lemma}[\cite{BS83}]\label{lem:baur-strassen}
  Let $\Phi$ be an arithmetic circuit of size $s$ and fan-in $2$ that computes a polynomial $f\in \F[X]$. Then, there is a multi-output circuit  of size $O(s)$ computing all first order derivatives of $f$.
\end{lemma}

Note that this immediately implies that any circuit computing $f = x_1^{d+1} + \dots + x_n^{d+1}$ requires size $\Omega(d\log n)$ as claimed by Theorem~\ref{thm:baur-strassen}. 


\subsubsection{Computing all first order derivatives simultaneously}

Since we are working with fan-in $2$ circuits, the number of edges is at most twice the size. Hence let $s$ denote the number of edges in the circuit $\Phi$, and we shall prove by induction that all first order derivatives of $\Phi$ can be computed by a circuit of size at most $5s$. Pick a non-leaf node $v$ in the circuit $\Phi$ closest to the leaves with both its children being variables, and say $x_1$ and $x_2$ are the variables feeding into $v$. In other words, $v = x_1 \odot x_2$ where $\odot$ is either $+$ or $\times$.

Let $\Phi'$ be the circuit obtained by deleting the two edges feeding into $v$, and replacing $v$ by a new variable. Hence, $\Phi'$ computes a polynomial $f' \in \F[X\union \inbrace{v}]$ and has at most $(s-1)$ edges. By induction on the size, we can assume that there is a circuit $\mathbb{D}(\Phi')$ consisting of at most $5(s-1)$ edges that computes all the first order derivatives of $f'$.

Observe that since $f'\mid_{(v = x_1 \odot x_2)} = f(\vecx)$,  we have that 
$$
\parderiv{f}{x_i} \spaced{=}\inparen{\parderiv{f'}{x_i}}_{v = x_1 \odot x_2} \quad+\quad  \inparen{\parderiv{f'}{v}}_{v = x_1 \odot x_2}\inparen{\parderiv{(x_1 \odot x_2)}{x_i}}.
$$

Hence, if $v = x_1 + x_2$ then
\begin{eqnarray*}
  \parderiv{f}{x_1} & = & \inparen{\parderiv{f'}{x_1}}_{v=x_1 + x_2} +\quad \inparen{\parderiv{f'}{v}}_{v = x_1 + x_2}\\
  \parderiv{f}{x_2} & = & \inparen{\parderiv{f'}{x_2}}_{v=x_1 + x_2} +\quad \inparen{\parderiv{f'}{v}}_{v = x_1 + x_2}\\
  \parderiv{f}{x_i} & = & \inparen{\parderiv{f'}{x_i}}_{v=x_1 + x_2} \qquad\text{for $i>2$}.
\end{eqnarray*}
If $v = x_1 \cdot x_2$, then
\begin{eqnarray*}
  \parderiv{f}{x_1} & = & \inparen{\parderiv{f'}{x_1}}_{v=x_1 \cdot x_2} + \inparen{\parderiv{f'}{v}}_{v = x_1 \cdot x_2} \cdot x_2\\
  \parderiv{f}{x_2} & = & \inparen{\parderiv{f'}{x_2}}_{v=x_1 \cdot x_2} + \inparen{\parderiv{f'}{v}}_{v = x_1\cdot x_2}\cdot x_1\\
  \parderiv{f}{x_i} & = & \inparen{\parderiv{f'}{x_i}}_{v=x_1 \cdot x_2} \qquad\text{for $i>2$}.
\end{eqnarray*}

Hence, by adding at most $5$ additional edges to $\mathbb{D}(\Phi')$, we can construct $\mathbb{D}(\Phi)$ and hence size of $\mathbb{D}(\Phi)$ is at most $5s$. \qed (Lemma~\ref{lem:baur-strassen})



%%% LOCAL Variables: 
%%% mode: latex
%%% TeX-master: "lowerbounds"
%%% End: 
