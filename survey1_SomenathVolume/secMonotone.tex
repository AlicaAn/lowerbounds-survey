\section{Lower bounds for monotone circuits}

This section would present a slight generalization 
	of a lower bound by Jerrum and Snir~\cite{js82}. 
	To motivate our presentation here, let us first 
	assume that the underlying field is $\RR$, the 
	field of real numbers. A monotone circuit over 
	$\RR$ is a circuit having $+, \times$ gates in 
	which all the field constants are {\em non-negative} 
	real numbers. Such a circuit can compute any 
	polynomial $f$ over $\RR$ all of whose 
	coefficients are nonnegative real numbers, such as 
	for example the permanent. It is then natural to ask 
	whether there are small monotone circuits over $\RR$ 
	computing the permanent. Jerrum and Snir \cite{js82}
	obtained an exponential lower bound on the size 
	of monotone circuits over $\RR$ computing the 
	permanent. Note that this	definition of monotone 
	circuits is valid only over $\RR$ (actually more 
	generally over ordered fields but not over say finite 
	fields) and such circuits can only compute polynomials 
	with non-negative coefficients. Here we will present 
	Jerrum and Snir's argument in a slightly more 
	generalized form such that the circuit model makes 
	sense over any field $\FF$ and is complete, i.e. 
	can compute any polynomial over $\FF$. Let us first 
	explain the motivation behind the generalized circuit
	model that we present here. Observe that in any monotone 
	circuit over $\RR$, there is no cancellation as there 
	are no negative coefficients. Formally, for a node $v$ 
	in our circuits let us denote by $f_{v}$ the polynomial 
	computed at that node. For a polynomial $f$ let us denote 
	by $\Mon(f)$ the set of monomials having a nonzero 
	coefficient in the polynomial $f$. 
		\begin{enumerate}
		
			\item If $w = u + v$ then 
				$$ \Mon(f_{w}) = \Mon(f_{u}) \cup \Mon(f_{v}). $$
			
			\item If $w = u \times v$ then
				$$ \Mon(f_{w}) = \Mon(f_{u}) \cdot \Mon(f_{v}) \eqdef \setdef{m_1\cdot m_2}{m_1 \in \Mon(f_{u}), m_2\in \Mon(f_{v})}. $$
			
		\end{enumerate}
	This means that for any node $v$ in a monote circuit over 
	$\RR$ one can determine $\Mon(f_{v})$ in a very syntactic 
	manner starting from the leaf nodes. Let us make precise 
	this syntactic computation that we have in mind.
	
	\begin{definition}[Formal Monomials.]
  Let $\Phi$ be an arithmetic circuit. The \emph{formal monomials} at
  any node $v\in \Phi$, which shall be denoted by $\FM(v)$, shall 
  be inductively defined as follows:
	  \begin{quote}
	    If $v$ is a leaf labelled by a variable $x_i$, then $\FM(v) =
	    \inbrace{x_i}$. If it is labelled by a constant, then $\FM(v) =
	    \inbrace{1}$.
	    
	    If $v = v_1 + v_2$, then $\FM(v) = \FM(v_1) \union \FM(v_2)$. 
	
	    If $v = v_1 \times v_2$, then 
            \begin{eqnarray*}
              \FM(v) &=& \FM(v_1)\cdot \FM(v_2)\\
              &\eqdef& \setdef{m_1\cdot m_2}{m_1 \in \FM(v_1), m_2\in \FM(v_2)}.
            \end{eqnarray*}
          \end{quote}
	\end{definition}
	
\noindent Note that for any node $v$ in any circuit 
	we have $\Mon(f_{v}) \subseteq \FM(v)$ but in a 
	monotone circuit over $\RR$ this containment is in 
	fact an equality at every node. This motivates our 
	definition of a slightly more general notion of a 
	monotone circuit as follows. 

	
\begin{definition}[Monotone circuits]
  A circuit $C$ is said to be
  \emph{syntactically monotone}
   (simply monotone for short) if $\Mon(f_{v}) = \FM(v)$ 
   for every node $v$ in $C$.
\end{definition}
	
	
	
The main theorem of this section is the following: 

\begin{theorem}[\cite{js82}]\label{thm:monotone-circuit-lbs}
	Over any field $\FF$, any syntactically monotone circuit 
	$C$ computing $\Det_n$ or $\Perm_n$ must have
  size at least $2^{\Omega(n)}$.
\end{theorem}

The proof of this theorem is relatively short assuming the
	following structural result (which is present in standard
	depth-reduction proofs \cite{vsbr83,ajmv98}).

\begin{lemma}\label{lem:vsbr-two-thirds}
  Let $f$ be a degree $d$ polynomial computed by a monotone circuit of
  size $s$. Then, $f$ can be written of the form $f = \sum_{i=1}^s f_i
  \cdot g_i$ where the $f_i$'s and $g_i$'s satisfy the following
  properties.
\begin{enumerate}
\item For each $i\in [s]$, we have $\frac{d}{3} < \deg{g_i} \leq
  \frac{2d}{3}$.
\item For each $i$, we have $\FM(f_i)\cdot \FM(g_i) \subseteq \FM(f)$.
\end{enumerate}
\end{lemma}

We shall defer this lemma to the end of the section and first see how
this would imply Theorem~\ref{thm:monotone-circuit-lbs}. The complexity measure $\Gamma(f)$ in this case is just the number of monomials in $f$, but it is the above \emph{normal form} that is crucial in the lower bound. 

\begin{proofof}{Theorem~\ref{thm:monotone-circuit-lbs}}
Suppose $\Phi$ is a circuit of size $s$ that computes $\Det_n$. Then by Lemma~\ref{lem:vsbr-two-thirds}, 
$$
\Det_n \quad=\quad \sum_{i=1}^s f_i \cdot g_i
$$
with $\FM(f_i)\cdot \FM(g_i) \subseteq \FM(\Det_n)$. The building blocks are terms of the form $T = f\cdot g$, where $\FM(f)  \cdot \FM(g) \subseteq \FM(\Det_n)$. \\

Since all the monomials in $\Det_n$ are products of variables from
distinct columns and rows, the rows (and columns) containing the
variables $f$ depends on is disjoint from the rows (and columns)
containing variables that $g$ depends on. Hence, there exists sets of indices $A,B
\subseteq [n]$ such that $f$ depends only on $\setdef{x_{jk}}{j\in
  A, k\in B}$ and $g$ depends only on
$\setdef{x_{jk}}{j\in\overline{A}, k\in
  \overline{B}}$.

Further, since $\Det_n$ is a homogeneous polynomial of degree $n$, we also have that both $f$ and $g$ must be homogeneous as well. Also as all monomials of $g$ using distinct row and column indices from $\overline{A}$ and $\overline{B}$ respectively, we see that $\deg g = |\overline{A}| = |\overline{B}|$ and $\deg f = |A| = |B|$. Using Lemma~\ref{lem:vsbr-two-thirds}, let $|A| = \alpha n$ for some $\frac{1}{3}\leq \alpha \leq \frac{2}{3}$. This implies that $\Gamma(f)\leq (\alpha n)!$, and $\Gamma(g)\leq ((1-\alpha)n)!$ and hence
$$\Gamma(f\cdot g) \quad \leq\quad (\alpha n)! ((1-\alpha)n)! \spaced{\leq} \frac{n!}{\binom{n}{n/3}} $$
as $\frac{1}{3}\leq \alpha \leq \frac{2}{3}$.
Also, $\Gamma$ is clearly sub-additive and we have $$\Gamma(f_1g_1 + \dots + f_s g_s) \spaced{\leq} s \cdot \frac{n!}{\binom{n}{n/3}}.$$
Since $\Gamma(\Det_n) = n!$, this forces $s \geq
\binom{n}{n/3} = 2^{\Omega(n)}$.
\end{proofof}

We only need to prove Lemma~\ref{lem:vsbr-two-thirds} now.

\subsection{Proof of Lemma~\ref{lem:vsbr-two-thirds}}

Without loss of generality, assume that the circuit $\Phi$ is
homogeneous\footnote{It is a forklore result that any circuit can be \emph{homogenized} with just a polynomial blow-up in size. Further, this process also preserves monotonicity of the circuit. A proof of this may be seen in \cite{sy}.}, and consists of alternating layers of $+$ and $\times$
gates. Also, assume that all $\times$ gates have fan-in two, and
orient the two children such that the formal degree of the left child
is at least as large as the formal degree of the right child. Such
circuits are also called \emph{left-heavy} circuits.

\begin{definition}[Proof tree]\label{defn:prooftree}
A \emph{proof tree} of an arithmetic circuit $\Phi$ is a sub-circuit $\Phi'$ such that
\begin{itemize}
\item The root of $\Phi$ is in $\Phi'$
\item If a multiplication gate with $v = v_1\times v_2 \in \Phi'$, then $v_1$ and $v_2$ are in $\Phi'$ as well.
\item If an addition gate $v = v_1 + \dots + v_s \in \Phi'$, then
  exactly one $v_i$ is in $\Phi'$.
\end{itemize}
Such a sub-circuit $\Phi'$, represented as a tree (duplicating nodes if
required), shall be called a \emph{proof tree} of $\Phi$.
\end{definition}

\newcommand{\PF}{\textsc{ProofTrees}}

Let $\PF(\Phi)$ denote the set of all proof trees of $\Phi$. It is
easy to see that any proof tree of $\Phi$ computes a monomial over the
variables. Further, if $\Phi$ was a monotone circuit computing a polynomial
$f$, then every proof tree computes a monomial in $f$. Therefore,
$$
f\quad=\quad \sum_{\Phi' \in \PF(\Phi)} [\Phi']
$$
where $[\Phi']$ denotes the monomial computed by $\Phi'$. Of course, the number of proof trees is exponential and the above
expression is huge. However, we could use a divide-and-conquer
approach to the above equation using the following lemma. 

\begin{lemma}\label{lem:brent-two-thirds}
  Let $\Phi'$ be a left-heavy formula of formal degree $d$. Then there
  is a node $v$ on the left-most path of $\Phi'$ such that
  $\frac{d}{3}\leq \deg(v)< \frac{2d}{3}$.
\end{lemma}
\begin{proof}
  Pick the lowest node on the left-most path that has degree at least
  $\frac{2d}{3}$. Then, its left child must be a node of degree less
  than $\frac{2d}{3}$, and also at least $\frac{d}{3}$ (because the
  formula is left-heavy). 
\end{proof}

For any proof tree $\Phi'$ and a node $v$ on its left-most path,
define $[\Phi':v]$ to be the output polynomial of the proof tree 
obtained by replacing the node $v$ on the left-most path by $1$. 
If $v$ does not occur on the
left-most path of $\Phi'$, define $[\Phi':v]$ to be $0$. We will 
denote the polynomial computed at a node $v$ by $f_{v}$. Then, the
above equation can now be re-written as:
\begin{eqnarray*}
f & = & \sum_{\Phi' \in \PF(\Phi)} [\Phi']\\
  & = & \sum_{\substack{v\in \Phi\\ \frac{d}{3}\leq \deg{v} < \frac{2d}{3}}} f_{v} \cdot \inparen{ \sum_{\Phi'\in \PF(\Phi)} [\Phi':v]}\\
  & = & \sum_{\substack{v\in \Phi\\ \frac{d}{3}\leq \deg{v} < \frac{2d}{3}}} f_{v} \cdot g_v\quad\quad\text{where $g_v = \sum_{\Phi'\in \PF(\Phi)} [\Phi':v]$} .\\
\end{eqnarray*}
Since $\frac{d}{3} \leq \deg{v} < \frac{2d}{3}$, we also have that
$\frac{d}{3} < \deg{g_v} \leq \frac{2d}{3}$ and the last equation is
what was required by Lemma~\ref{lem:vsbr-two-thirds}.\qed


%%% Local Variables: 
%%% mode: latex
%%% TeX-master: "lowerbounds_birk"
%%% End: 
