
%% \newenvironment{problem}[1]{\stepcounter{problem}
%% \paragraph{Problem \theproblem. #1}

%% }


\newtheorem{theorem}{Theorem}
\newtheorem{theorem*}{Theorem}
\newtheorem{axiom}{Axiom}
\newtheorem{corollary}[theorem]{Corollary}
\newtheorem{lemma}[theorem]{Lemma}
\def\lemmaautorefname{Lemma}
\newtheorem{observation}[theorem]{Observation}
\newtheorem{conjecture}[theorem]{Conjecture}
\newtheorem{proposition}[theorem]{Proposition}
\newtheorem{definition}[theorem]{Definition}
\newtheorem{claim}[theorem]{Claim}
\newtheorem{remark}[theorem]{Remark}
\newtheorem*{remark*}{Remark}
\newtheorem*{nclaim}{Claim}
\newtheorem{fact}[]{Fact}
\newtheorem{subclaim}[theorem]{Subclaim}
\newtheorem{problem}[]{Problem}
\newtheorem{openproblem}[]{Open Problem}

\newenvironment{myproof}[1]%
{\vspace{1ex}\noindent{\em Proof.}\hspace{0.5em}\def\myproof@name{#1}}%
{\hfill{\tiny \qed\ (\myproof@name)}\vspace{1ex}}


%Useful for recalling theorems. (hat-tip texexchange)
\makeatletter
\newtheorem*{rep@theorem}{\rep@title}
\newcommand{\newreptheorem}[2]{%
\newenvironment{rep#1}[1]{%
 \def\rep@title{#2 \ref{##1} (restated)}%
 \begin{rep@theorem}}%
 {\end{rep@theorem}}}
\makeatother

\newreptheorem{theorem}{Theorem}
\newreptheorem{lemma}{Lemma}
\newreptheorem{conjecture}{Conjecture}
\newreptheorem{problem}{Problem}

\newenvironment{proof-sketch}{\medskip\noindent{\em Sketch of Proof.}\hspace*{1em}}{\qed\bigskip}
\newenvironment{proof-attempt}{\medskip\noindent{\em Proof attempt.}\hspace*{1em}}{\bigskip}

\newenvironment{proofof}[1]{\medskip\noindent\emph{Proof of #1.}\hspace*{1em}}{\qed\bigskip}



\newcommand{\inparen }[1]{\left(#1\right)}             %\inparen{x+y}  is (x+y)
\newcommand{\inbrace }[1]{\left\{#1\right\}}           %\inbrace{x+y}  is {x+y}
\newcommand{\insquare}[1]{\left[#1\right]}             %\insquar{x+y}  is [x+y]
\newcommand{\inangle }[1]{\left\langle#1\right\rangle} %\inangle{A}    is <A>

\newcommand{\abs}[1]{\left|#1\right|}                  %\abs{x}        is |x|
\newcommand{\norm}[1]{\left\Vert#1\right\Vert}         %\norm{x}       is ||x||

\newcommand{\fspan}[1]{\F\text{-span}\inbrace{#1}}

\newcommand{\union}{\cup}
\newcommand{\Union}{\bigcup}
\newcommand{\intersection}{\cap}
\newcommand{\Intersection}{\bigcap}

\newcommand{\ceil}[1]{\lceil #1 \rceil}
\newcommand{\floor}[1]{\lfloor #1 \rfloor}


\newcommand{\eqdef}{\stackrel{\text{def}}{=}}
\newcommand{\setdef}[2]{\inbrace{{#1}\ : \ {#2}}}      % E.g: \setdef{x}{f(x) = 0}
\newcommand{\set}[1]{\inbrace{#1}}
\newcommand{\innerproduct}[2]{\left\langle{#1},{#2}\right\rangle} %\innerproduct{x}{y} is <x,y>.
\newcommand{\zo}{\inbrace{0,1}}                        % Well just something that is used often!
\newcommand{\parderiv}[2]{\frac{\partial #1}{\partial #2}}
\newcommand{\pderiv}[2]{\partial_{#2}\inparen{#1}}
\newcommand{\zof}[2]{\inbrace{0,1}^{#1}\longrightarrow \inbrace{0,1}^{#2}}


% Commonly used blackboard letters
\newcommand{\FF}{\mathbb{F}}
\newcommand{\F}{\mathbb{F}}
\newcommand{\N}{\mathbb{N}}
\newcommand{\Q}{\mathbb{Q}}
\newcommand{\Z}{\mathbb{Z}}
\newcommand{\R}{\mathbb{R}}
\newcommand{\C}{\mathbb{C}}
\newcommand{\RR}{\mathbb{R}}
\newcommand{\E}{\mathbb{E}}


\newcommand{\zigzag}{\textcircled{z}}  % for the zig-zag product


% \newcommand{\char}{\textrm{char}}
% \newcommand{\rank}{\textrm{rank}}
% \newcommand{\dim}{\textrm{dim}}



%% accented words
\newcommand{\Hastad}{H{\aa}stad }
\newcommand{\Godel}{G\"{o}del }
\newcommand{\Mobius}{M\"{o}bius }
\newcommand{\Gauss}{Gau{\ss} }
\newcommand{\naive}{na\"{\i}ve }
\newcommand{\Naive}{Na\"{\i}ve }
\newcommand{\grobner}{gr\"{o}bner }
\newcommand{\bezout}{b\'{e}zout}
\newcommand{\Bezout}{B\'{e}zout}



\newcommand{\Det}{\mathsf{Det}}
\newcommand{\Perm}{\mathsf{Perm}}
\newcommand{\ESym}{\mathrm{Esym}}
\newcommand{\PSym}{\mathrm{Pow}}

%% Bold letters
\newcommand{\veca}{\mathbf{a}}
\newcommand{\vecb}{\mathbf{b}}
\newcommand{\vecc}{\mathbf{c}}
\newcommand{\vecd}{\mathbf{d}}
\newcommand{\vece}{\mathbf{e}}
\newcommand{\vecf}{\mathbf{f}}
\newcommand{\vecg}{\mathbf{g}}
\newcommand{\vech}{\mathbf{h}}
\newcommand{\veci}{\mathbf{i}}
\newcommand{\vecj}{\mathbf{j}}
\newcommand{\veck}{\mathbf{k}}
\newcommand{\vecl}{\mathbf{l}}
\newcommand{\vecm}{\mathbf{m}}
\newcommand{\vecn}{\mathbf{n}}
\newcommand{\veco}{\mathbf{o}}
\newcommand{\vecp}{\mathbf{p}}
\newcommand{\vecq}{\mathbf{q}}
\newcommand{\vecr}{\mathbf{r}}
\newcommand{\vecs}{\mathbf{s}}
\newcommand{\vect}{\mathbf{t}}
\newcommand{\vecu}{\mathbf{u}}
\newcommand{\vecv}{\mathbf{v}}
\newcommand{\vecw}{\mathbf{w}}
\newcommand{\vecx}{\mathbf{x}}
\newcommand{\vecy}{\mathbf{y}}
\newcommand{\vecz}{\mathbf{z}}

\newcommand{\spaced}[1]{\quad #1 \quad}
