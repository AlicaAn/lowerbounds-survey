\chapter{Lower bounds for depth-3 circuits}\label{chap:d3SW}

In this chapter, we shall see the lower bound of Shpilka and Wigderson~\cite{sw2001} for non-homogeneous depth-$3$ circuits over arbitrary fields.
The main theorem of this section would be the following \emph{quadratic} lower bound. 

\begin{theorem}[\cite{sw2001}]\label{thm:SW-SPS-main-thm}
Any $\SPS$ circuit that computes the polynomial $\Det_n$ (or $\Perm_n$) must have at least $\Omega\pfrac{n^4}{\log n}$ wires. 
\end{theorem}

Note that the number of variables in $\Det_n$ is $n^2$ and hence the above lower bound is a quadratic lower bound in the number of variables.
Until recently, this was the best lower bound we knew for the class of general $\Sigma\Pi\Sigma$ circuits but a very recent result of Kayal, Saha and Tavenas~\cite{kst16} has improved this to an almost cubic lower bound (for a different explicit family of polynomials) which we shall see at a later point.
This section however shall focus on the proof of the above theorem.

\section{Revisiting the hom. $\SPS$ lower bound of \cite{nw1997}}

From \autoref{obs:low-rank-sps-rank}, if $T = \ell_1 \cdots \ell_d$ then for every $0\leq k \leq d$ we have
\[
\dim \partial^{=k}(T) \spaced{\leq} \binom{d}{k}. 
\]
For a polynomial such as the $\Det_n$, we know that $\partial^{=k}(\Det_n) = \binom{n}{k}^2$. Let us first get a sense of what is the best $k$ to choose for proving lower bounds for $\Sigma\Pi^{[d]}\Sigma$ circuits. We recall some bounds on binomial coefficients from \autoref{chap:binom-estimates}. 

\begin{lemmawp}For any $n\geq k \geq 0$, 
\[
\pfrac{n}{k}^k \spaced{\leq} \binom{n}{k} \spaced{\leq} \pfrac{ne}{k}^k\qedhere
\]
\end{lemmawp}

\begin{lemma}[\cite{nw1997}]\label{NW:d3hom-mainlemma} For any $d \geq 0$, any $\Sigma\Pi^{[d]}\Sigma$ circuit computing $\Det_n$ must have size $\exp\inparen{\Omega\pfrac{n^2}{d}}$. 
\end{lemma}
\begin{proof}
As just mentioned, if $C = T_1 + \cdots + T_s$ where each $T_i$ is a product of at most $d$ linear polynomials, then 
\[
\dim(\partial^{=k}(C)) \spaced{\leq } s \cdot \binom{d}{k}.
\]
On the other hand, 
\[
\dim(\partial^{=k}(\Det_n)) \spaced{\geq } \binom{n}{k}^2.
\]
Therefore, we get a lower bound of 
\begin{eqnarray*}
s & \geq & \frac{\binom{n}{k}^2}{\binom{d}{k}}\\
  & \geq & \pfrac{n}{k}^{2k} / \pfrac{de}{k}^k \spaced{=} \pfrac{n^2}{e\cdot dk}^k
\end{eqnarray*}
Thus, by choosing $k = \pfrac{n^2}{2ed}$, we get a lower bound of $2^k = \exp\inparen{\Omega\pfrac{n^2}{d}}$. 
\end{proof}


\section{The win-win proof of \cite{sw2001}}

\begin{proof}[Proof of \autoref{thm:SW-SPS-main-thm}]
 Say $\Det_n$ is computed by $C = T_1 + \cdots + T_s$ where each $T_i$ is a product of linear forms.
We will prove by induction that $s\geq \frac{n^4}{100\log n} + 1$. Starting with $C$, we shall modify the circuit to a new circuit $C'$ that computes $\Det_{n-1}$ (up to renaming variables). 

We shall keep in mind a threshold $d$ that would be set shortly (spoiler: $d= n^2/20\log n)$).
By \autoref{NW:d3hom-mainlemma} we know that if each $T_i$ is a product of at most $d$ linear polynomials, we have a lower bound of $\exp(\Omega(n^2/d))$. We shall say a variable $x_{ij}$ is \emph{$d$-good} in $C$ if each $T_i$ that depends on $x_{ij}$ has degree at most $d$. \\

{\bf Case 1:} There is a variable in the first row $x_{1j}$ that is $d$-good. 

\medskip

\noindent 
In this setting, consider the polynomial $C' = C_{(x_{1i} = 1)} - C_{(x_{1i} = 0)}$.
As far as $\Det_n$ is concerned, this is equivalent to taking the derivative with respect to $x_{ij}$ and hence the polynomial is the corresponding $(n-1)\times (n-1)$ minor.
The circuit $C'$ on the other hand now is a $\Sigma\Pi^{[d]}\Sigma$ circuit with top fan-in at most $2s$, as all $T_i$s in $C$ that do not depend on $x_{ij}$ would now be eliminated.
By \autoref{NW:d3hom-mainlemma}, this implies that $2s > \exp(\Omega((n-1)^2/d))$ which is certainly bigger than $(n^4 / \log n)$ if $d = n^2/20\log n$.\\

{\bf Case 2:} There is \emph{no} variable in the first row that is $d$-good. 

\medskip

\noindent 
We know that the variable $x_{12}$ is not $d$-good.
This means that there is some $T_i$ of degree more than $d$ which has a factor of the form $(x_{12} - \ell)$.
We shall now set $x_{12} = \ell$ in $C$ and this eliminates the gate $T_i$ altogether and thereby reducing the size by at least $d$ wires. 

Now move on to $x_{13}$. It is possible now that after eliminating $T_i$ in the previous step, we may now have $x_{13}$ to be $d$-good. Then we reduce to Case 1 and we are done. Otherwise, there is some $T_j$ of degree more than $d$ with a factor of the form $(x_{13} - \ell')$. Once again, we set $x_{13} = \ell'$ to eliminate $T_j$ and thereby reduce the size of the circuit by $d$. \\

Repeating this process for $\set{x_{12}, \cdots, x_{1n}}$, we either reach Case 1 (in which case we are done), or we have thereby reduced the circuit by at least $d(n-1)$ wires. So far, we have only messed with the variables $\set{x_{12},\cdots, x_{1n}}$ but this is not an issue because we can further set $x_{11} = 1$ and $x_{21} = x_{31} = \cdots = x_{n1} = 0$. Therefore, the polynomial thus computed has to be the minor with respect to $x_{11}$ which is just $\Det_{n-1}$ up to renaming variables. But then,
\[
\mathrm{size}(C) \spaced{\geq} \mathrm{size}(C') \;+\; \pfrac{(n-1)^2 n^2}{20\log n}
\]
Hence, it follows by induction that $\mathrm{size}(C) \geq \pfrac{n^4}{100\log n} + 1$.
\end{proof}

%%% Local Variables: 
%%% mode: latex
%%% TeX-master: "fancymain"
%%% End: 
