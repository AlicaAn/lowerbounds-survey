\chapter{Tensor rank and formula lower bounds}\label{chap:tensorrk}

In this chapter, we will establish an apriori surprising connection between lower bounds for homogeneous arithmetic formula and constructing explicit tensors of full rank.
The chapter is based on a result of Raz~\cite{raz10}.

\section{Tensors}

Tensors are natural \emph{higher dimensional} analogues of matrices.
A matrices is nothing but a two dimensional array filled in with numbers from some underlying field.
A tensor is a higher dimensional version of this, where we have an $d$-dimensional cuboid filled with numbers.
A tensor $T$ is a map of the form
\[
T: [m_1] \times \cdots \times [m_d] \longrightarrow \F
\]
the same way an $m\times n$ matrix a map from $[m] \times [n] \rightarrow \F$.
However, if one wants to understand a tensor more functionally (similarly to how it is useful to think of matrices as linear transformations on vector spaces), it is more natural to extend this definition linearly as follows.

\begin{definition}[Tensor]\label{defn:tensor}
A tensor $T$ is a map of the form 
\[
T: V_1 \times \cdots \times V_d \longrightarrow \F
\]
where each $V_i$ is a vector space over $\F$, of say dimension $m_i$, which is linear in every coordinate i.e.
\[
T(\vecv_1, \ldots,\alpha\vecv_i + \beta\vecv_i', \ldots, \vecv_d) = \alpha T(\vecv_1, \ldots,\vecv_i, \ldots, \vecv_d) + \beta T(\vecv_1, \ldots,\vecv_i', \ldots, \vecv_d).
\]
The parameter $r$ is called the \emph{order} of the tensor, and we say that the \emph{shape of $T$} is $m_1 \times \cdots \times m_d$. 
\end{definition}
Since a tensor is linear in every coordinate, it suffices to specify the image of $T$ at the basis of $V_1 \times \cdots \times V_d$ and extend it linearly.
So a tensor can indeed be thought of as filling up a $d$-dimensional array of shape $[m_1]\times \cdots \times [m_d]$ by field elements, the same way an $m\times n$ matrix is specified by an $m\times n$ array filled up with field elements.
Indeed, a matrix is nothing but an order-$2$ tensor.

It would sometimes be useful to switch between the two notions of thinking of a tensor as a multilinear map from $V_1 \times \cdots \times V_d$ to $\F$ and thinking a tensor as just a map from $[m_1] \times \cdots \times [m_d]$.
So throughout this chapter, we shall fix a basis $\set{\vece_{i1},\cdots, \vece_{im_i}}$ for $V_i$ and when we shall use $T[j_1,\cdots, j_d]$ to really denote $T(\vece_{1j_1},\cdots, \vece_{dj_d})$.

\subsection{Tensors as polynomials}

In our setting, it would be useful to think of tensors as a restricted form of multilinear polynomials that are called \emph{set-multilinear polynomials}.

\begin{definition}[Set-multilinear polynomials]\label{defn:set-multilinear}
  Let $\vecx = \vecx_1 \sqcup \cdots \sqcup \vecx_d$ be a partition of variables and let $|\vecx_i| = m_i$.
A polynomial $f(\vecx)$ is said to be \emph{set-multilinear} with respect to the above partition if every monomial $m$ in $f$ satisfies $\abs{m \intersection X_i} = 1$ for all $i \in [d]$.
\end{definition}
In other words, each monomial in $f$ picks up at most one variable from each part in the partition. It is easy to see that many natural polynomials such as $\Det$ or $\Perm$ or $\NW$ are all set-multilinear for an appropriate partition of variables. 

\begin{observation}\label{obs:tensor-to-sml}
  For any tensor $T$ of shape $[m_1] \times \cdots \times [m_d]$, we can associate a set-multilinear polynomial $f(\vecx)$ with $\vecx = \vecx_1 \sqcup \cdots \sqcup \vecx_d$ and $\vecx_i = \set{x_{i1}, \cdots, x_{im_i}}$ as
\begin{equation}\label{eqn:tensor-sml}
f(\vecx) \spaced{=} \sum_{\substack{1 \leq i_j \leq m_j\\ \forall j \in [d]}}  T(i_1,\cdots, i_d) \cdot x_{1i_1}\cdots x_{di_d}.
\end{equation}
\end{observation}

Another representation is to just use a single variable $x_j$ for a part $\vecx_j$ and use higher powers.
That way, we can associate the following polynomial with a tensor $T$:
\begin{equation}\label{eqn:tensor-to-prod-univariates}
f(x_1,\cdots, x_d) \spaced{=} \sum_{\substack{1 \leq i_j \leq m_j\\ \forall j \in [d]}}  T(i_1,\cdots, i_d) \cdot x_{1}^{i_1}\cdots x_{d}^{i_d}.
\end{equation}
The same also holds in the other direction where we can interpret any set-multilinear polynomial as an appropriate tensor. 


\subsection{Rank of a tensor}

Just like any matrix has a natural definition of \emph{rank}, there is an analogue for tensors as well.
The rank of a matrix $M$ can be defined as the smallest $r$ for which $M$ can be written as a sum of $r$ matrices of rank $1$.
A rank-$1$ matrix is just a matrix of the form $\vecu \vecv^T$ where the $(i,j)$-th entry is $u_iv_j$.
We shall abuse\footnote{it is abuse because it is really a tensor product of $\vecu$ and $\vecv^T$.}
notation and use $\vecu \otimes \vecv$ to denote the order-$2$ tensor $T$ where $T[i,j] = u_i v_j$.
This naturally generalizes to higher order as well.

\begin{definition}[Elementary tensors, and tensor rank]
  For $\vecv_1\in V_1, \cdots, \vecv_d \in V_d$, define the tensor $\vecv_1 \otimes \cdots \otimes \vecv_d$ to be the tensor $E$ given by
\[
E[j_1,\cdots, j_d] = (\vecv_1)_{j_1} \cdots (\vecv_d)_{j_d}.
\]
We shall call such tensors as \emph{elementary tensors} or \emph{rank-$1$ tensors}. 

For an arbitrary tensor $T$, the \emph{tensor rank of $T$}, denoted by $\rank(T)$, is the smallest $r$ such that $T$ can be expressed as a sum of $r$ elementary tensors. 
\end{definition}

\noindent 
What do elementary tensors look like as polynomials?
Let us consider the \emph{set-multilinear} polynomial setting as in \eqref{eqn:tensor-sml}.
It is easy to see that a rank-$1$ tensor is precisely a \emph{set-multilinear} product of linear forms such as
\[
f(\vecx) \spaced{=} \ell_1(\vecx_1) \cdots \ell_d(\vecx_d)
\]
where each $\ell_i(\vecx_i)$ is a linear form in the variables in $\vecx_i$. \\

In the setting of \eqref{eqn:tensor-to-prod-univariates}, it is easy to see that a rank-$1$ tensor is precisely a \emph{product of univariates} such as
\[
f(\vecx) \spaced{=} f_1(x_1)\cdots f_d(x_d).
\]
\noindent
Hence the following three questions are equivalent:
\begin{itemize}
\setlength\itemsep{0em}
\item Given a tensor $T$, find its rank. 
\item Given a set-multilinear polynomial $f$, find the smallest set-multilinear $\SPS$ circuit computing it. 
\item Given a polynomial $f$, find the smallest expression as a sum of product of univariates.
\end{itemize}

However, unlike matrices, computing the rank of even an order-$3$ tensor is $\NP$-hard \cite{h90}.
But one could still ask if we can prove good upper or lower bounds for some specific tensors, or try to find a tensor with large rank.
But before that, let us look at some basic properties that tensor rank satisfies.

\subsection*{Properties of  tensor rank}

The following are a couple of basic properties that follows almost immediately from the definitions.

\begin{lemma}[Sub-additivity of tensor rank]\label{lem:tensor-subadditivity}
  Let $T_1$ and $T_2$ be two tensors of the same shape and order.
Then, if $T = T_1 + T_2$, then $\rank(T) \leq \rank(T_1) + \rank(T_2)$.
\end{lemma}

\begin{lemma}[Sub-multiplicativity of tensor rank]\label{lem:tensor-submultiplicativity}
  Let $T_1: V_1 \times \cdots \times V_{d_1} \rightarrow \F$ and $T_2: W_1 \times \cdots \times W_{d_2} \rightarrow \F$ be two tensors.
Then if $T = T_1 \otimes T_2$ given by
\[
T[i_1,\cdots, i_{d_1},j_1,\cdots, j_{d_2}] = T_1[i_1,\cdots, i_{d_1}] \cdot T_2[j_1,\cdots, j_{d_2}],
\]
then $\rank(T) \leq \rank(T_1) \cdot \rank(T_2)$. 
\end{lemma}

\begin{exercise}
  Show that in fact $\rank(T_1 \otimes T_2) = \rank(T_1) \cdot \rank(T_2)$.
\end{exercise}

\subsection{Upper bounds on tensor rank}

Let us consider an order-$d$ tensor $T$ of shape $n \times \cdots \times n$.
How large can $\rank(T)$ be?
One possible upper bound we could say is $n^d$.
Surely, the tensor is an $d$-dimensional array with just $n^d$ entries.
We can certainly write it as a sum of $n^d$ elementary tensors of the form $\vece_{j_1} \otimes \cdots \otimes \vece_{j_d}$.
So clearly $\rank(T) \leq n^d$.
But we can do a little better.
Consider the case when $d=2$, and we have an $n\times n$ matrix.
The bound on the rank is not $n^2$ but rather $n$.
This indicates that one should be able to do a bit better than $n^d$ for the general case.
Indeed we can.

\begin{lemma}\label{lem:tensor-rank-trivial-upperbound}
  Let $T$ be an order-$d$ tensor of shape $n\times \cdots \times n$.
Then, $\rank(T) \leq n^{d-1}$.
\end{lemma}
\begin{proof}
  Let us revisit the case when $d=2$, where we know an $n\times n$ matrix has rank at most $n$.
Interpreting this statement via \eqref{eqn:tensor-to-prod-univariates}, this implies that of $q(x_1,x_2)$ is a bi-variate with degree in each variable bounded by $n$, then $q$ can be written as
\[
q(x_1,x_2) = \sum_{i=1}^n g_i(x_1) h_i(x_2).
\]
Therefore, if $f(x_1,\cdots, x_d)$ is a polynomial with degree in each variable bounded by $n$, then we can write $f$ as
\begin{align*}
f(x_1,\cdots, x_d) &= \sum_{m \in \mathrm{Mon}\set{x_3,\cdots, x_d}} m \cdot q_m(x_1,x_2)\\
& = \sum_{m \in \mathrm{Mon}\set{x_3,\cdots, x_d}} m \cdot \inparen{\sum_{i=1}^n g_{m,i}(x_1) \cdot h_{m,i}(x_2)}
\end{align*}
which is a sum of product of univariates of top fan-in $n \cdot n^{d-2} = n^{d-1}$. 
\end{proof}

A counting argument would say that there do exist tensors of rank at least $n^{d-1}/d$ as each elementary tensor has $nd$ \emph{degrees of freedom} and an arbitrary tensor has $n^d$ \emph{degrees of freedom}.
One might think that the above upper bound of $n^{d-1}$ should be tight.
Bizarrely, it is not!
For example (cf.
\cite{p85}), the maximum rank of any tensor of shape $2\times 2 \times 2$ is $3$ and not $4$ as one might expect!
Tensor rank also behaves in some strange ways under \emph{limits} unlike the usual matrix rank.
But a big open question is to find explicit tensors of such large rank.

\begin{openproblem}
Can we find an explicit tensor $T: [n]^d \rightarrow \F$ of rank $n^{d(1 - o(1))}$?
\end{openproblem}

Raz's \cite{raz10} showed that in certain regimes, an answer to the above question would yield arithmetic formula lower bounds. 

\section{Tensor rank of small formulas}

From this section onwards, we shall assume the \emph{set-multilinear polynomial} interpretation of a tensor $T:[n]^d \rightarrow \F$ as described in~\eqref{eqn:tensor-sml}.
Hence our variables $\vecx$ is partitioned as $\vecx = \vecx_1 \sqcup \cdots \sqcup \vecx_d$ with $\abs{\vecx_i} = n$ for all $i\in [d]$.
The main motivating question would be the following:
\begin{quote}
If $f$ is a set-multilinear polynomial that is computed by a small formula, what can one say about its tensor rank?
\end{quote}

To begin with, let us restrict ourselves to certain structured formulas that in a sense \emph{respects} the partition defined. 

\begin{definition}[Set-multilinear formulas] A formula $\Phi$ is said
  to be a \emph{set-multilinear formula} if for every gate in the formula computes a set-multilinear polynomial.
\end{definition}

From the above definition, note that the set-multilinear formulas are a subclass of homogeneous formulas.
As in the multilinear setting, it is easy to see that set-multilinearity for formulas can be made a \emph{syntactic} restriction where each gate computes a tensor, with addition gates only adding ``alike'' tensors and multiplication gates multiplying disjoint tensors.

\begin{exercise}
  Show that set-multilinear formulas can, without loss of generality, be assumed to be syntactically set-multilinear formulas.
\end{exercise}

\noindent
An easier question to the one above would be the following:

\begin{quote}
  If $f$ is a set-multilinear polynomial that is computed by a small \emph{set-multilinear} formula, what can one say about its tensor rank?
\end{quote}

In the rest of this chapter, we shall prove the following result of Raz \cite{raz10}. 

\begin{theorem}[\cite{raz10}] \label{thm:raz-tensorrank-sml} Let
  $\Phi$ be a set-multilinear formula of size $s \leq n^c$ computing a polynomial $f(\vecx_1,\cdots, \vecx_d)$.
Then,
\[
\rank(f) \spaced{\leq} \frac{n^d}{n^{d/\exp(c)}}. 
\] 
\end{theorem}

In the setting when $d$ is small, Raz \cite{raz10} also showed that formulas can be converted to set-multilinear formulas with a modest cost. 

\begin{theorem}[\cite{raz10}]\label{thm:form-to-smlform}
  Suppose $d = O\pfrac{\log n}{\log\log n}$.
If $\Phi$ is a formula of size $s = \poly(n)$ that computes a set-multilinear polynomial $f(\vecx_1,\cdots, \vecx_d)$, then there is a \emph{set-multilinear} formula of $\poly(s)$ size that computes $f$ as well.
\end{theorem}

As a corollary, finding explicit tensors of almost full rank would imply super-polynomial formula lower bounds in the low-degree regime.

\begin{corollary}[\cite{raz10}]
  If $f(\vecx_1,\cdots, \vecx_d)$ is an explicit tensor of rank $n^{d(1- o(1))}$ with $\omega(1) = d = O\pfrac{\log n}{\log\log n}$, then any formula computing $f$ must be of super-polynomial size.
\end{corollary}

The above two theorems are of very different flavours and should really be thought of as two independent surprising results.
\autoref{thm:raz-tensorrank-sml} is a tensor-rank upper bound and \autoref{thm:form-to-smlform} is a structural result.
We shall first address \autoref{thm:raz-tensorrank-sml} in the next section and address \autoref{thm:form-to-smlform} after that.

\subsection{The tensor-rank upper-bound}

We shall now prove \autoref{thm:raz-tensorrank-sml}. The proof described here is not the original proof in \cite{raz10} but is an alternate proof by Mrinal Kumar and Saptharishi. 

\begin{proof}[Proof of \autoref{thm:raz-tensorrank-sml}]



For this we would need the slightly better depth reduction for homogeneous formulas (\autoref{thm:av-formulas}) of Saptharishi and Vinay \cite{saptharishivinay14}. We recall the statement here.
\begin{theorem*}[\autoref{thm:av-formulas}]
Let $f$ be a homogeneous $n$-variate degree $d$ polynomial computed by a size $s$ homogeneous formula. 
Then for any $0< t \leq d$, $f$ can be equivalently computed by a homogeneous $\Sigma\Pi^{[a]}\Sigma\Pi^{[t]}$ formula of top fan-in $s^{10(d/t)}$ where 
\[
a \geq \frac{1}{10}\pfrac{d}{t} \log t.
\]
\end{theorem*}
It is a fairly straightforward observation to see that the above theorem preserves multilinearity and set-multilinearity as well. We shall start with the set-multilinear formula $\Phi$ of size $s = n^c$ that computes the polynomial $f(\vecx_1,\cdots, \vecx_d)$ and apply \autoref{thm:av-formulas} for a suitable choice of $t$ (that shall be set shortly). 
Therefore we now have a set-multilinear expression of the form
\[
f \spaced{=} T_1 + \cdots + T_{s'}
\]
where $s' \leq s^{10(d/t)} = n^{10c(d/t)}$ and each $T_i = Q_{i1} \cdots Q_{ia_i}$ is a set-multilinear product. Let us fix one such term $T = Q_{1} \cdots Q_a$ and we know that this is a set-multilinear product with $a \geq \pfrac{d\log t}{10 t}$. Let $d_i = \deg(Q_i)$. By the sub-multiplicativity of tensor rank (\autoref{lem:tensor-submultiplicativity}) and the trivial upper bound (\autoref{lem:tensor-rank-trivial-upperbound}) we have
\begin{align*}
\rank(T) & \leq n^{d_1 - 1} \cdots n^{d_a - 1}\\
 & =  n^{d - a}\\
\implies \rank(f) & \leq s' \cdot n^{d-a} & \text{(~\autoref{lem:tensor-subadditivity})}\\
 & =  \pfrac{n^d}{n^{a - 10c(d/t)}}
\end{align*}
Let us focus on the exponent of $n$ in the denominator. Using the lower bound on $a$ from \autoref{thm:av-formulas}, we get
\begin{align*}
a - 10c(d/t) & \geq \frac{d\log t}{10t} - 10c\pfrac{d}{t}\\
& = \pfrac{d}{t} \inparen{\frac{\log t}{10} - 10c}\\
& = \pfrac{d}{2^{O(c)}}& \text{if we set $\frac{\log t}{10} = 11c$}\\
\implies \rank(f) & \leq \frac{n^d}{n^{d/\exp(c)}}
\end{align*}
which is what we set out to prove. 
\end{proof}

\subsection{Making formulas set-multilinear}

In this section we shall prove \autoref{thm:form-to-smlform}.
The proof would proceed in two steps, both of which are quite interested in their own right.
\begin{itemize}
\setlength\itemsep{0em}
\item {\bf Homogenization :} In the first step, we shall convert a
  formula that compute a homogeneous polynomial of degree $d$ in $n$ variables to a n homogeneous formula computing the same polynomial.
It would turn out that if $d = O(\log n)$, then this transformation only incurs a $\poly(n)$ blow-up in the size.
\item {\bf Set multilinearization :} In the second step, we show that
  for any homogeneous arithmetic formula that computes a set multilinear polynomial of degree $d$ in $n$ variables can be converted to a set-multilinear formula computing the same polynomial.
In this setting, if $d \leq O(\log n/\log \log n)$, the blow-up in the size of the formula will only be $\poly(n)$.
\end{itemize}

\subsubsection*{Homogenization of low-degree formulas}

\begin{lemma}[\cite{raz10}]\label{lem:formula-homogenization} Let $\Phi$ be a formula of size $s$ computing an $n$-variate homogeneous polynomial $f$ of degree $d$. Then, there is a \emph{homogeneous formula} $\Phi'$ computing $f$ of size at most $\poly\inparen{s,\binom{d+\log s}{d}}$. 

In particular, if $d = O(\log n)$ and $n = \poly(n)$ then we have $\mathrm{size}(\Phi') = \poly(n)$ as well. 
\end{lemma}
\begin{proof}
Recall that due to the depth reduction result of Brent and Spira (\autoref{lem:formula-depth-reduction}), we can assume without loss of generality that the fan-in of every gate in $\Phi$ at most $2$ and the depth of $\Phi$ is $O(\log s)$.
The construction of the new formula $\Phi'$ would have no surprises -- homogenize the formula $\Phi$ to obtain a circuit $C$ using the standard homogenization (\autoref{lem:homogenization}), and unravel the circuit to make it a formula in the most natural way. It is the analysis of the size that would give the lemma. 

\medskip

For every gate $v$ in $\Phi$, we have $d+1$ gates $(v, 0), (v, 2), \ldots, (v, d)$ in $C$.
Semantically, the polynomial computed at such a gate $(v, i)$ is the degree-$i$ homogeneous component of the polynomial computed at $v$ in $\Phi$. As in \autoref{lem:homogenization}, we shall connect these edges as follows:
\begin{eqnarray*}
v = u+w\quad\implies\quad (v, i) &=& (u, i) \spaced{+} (w,i)\quad\text{for all $i$}\\
v = u\times w\quad\implies\quad (v, i) &=& \sum_{j=0}^i (u,j)\cdot (w, {i-j}) \quad\text{for all $i$}
\end{eqnarray*}
Hence, we now have a homogeneous circuit $C$ that computes $f$ and has size at most $s' = O(sd^2)$. Furthermore, the depth of this circuit is at most twice the depth of the formula $\Phi$ which was $O(\log s)$. Hence $C$ is a homogeneous circuit of depth $O(\log s)$ and size $O(sd^2)$ computing $f$. 

To convert $C$ into a formula, we will do the natural operation of \emph{recomputing} a node whenever we need to reuse the computation. This is equivalent to duplicating every gate $(v,i)$ of $C$ as many times are there are paths from this gate to the root fo $C$.
Thus, in order to upper bound the size of the resulting formula, we will require an upper bound on the number of distinct paths from every gate $(v,i)$ of $C$ to its root.

\medskip

Currently, $C$ is a circuit because if $v$ is a product gate of $\Phi$ with children $u$ and $w$, then the out degree of $(u, j)$ and $(w, j)$ in $C$ could be more than $1$ as it \emph{contributes} to $(v,j')$ for all $j \leq j' \leq d$.
Hence, the resulting structure is a circuit and not a formula.
However, at sum gates, the out degree of the children continue to be $1$.
But this gives us a good understanding of the many paths from $(v,i)$ to the root in $C$.
Let $v\rightarrow v_1 \rightarrow \cdots \rightarrow v_r$ be the path from $v$ to the root ($=v_r$) in $\Phi$. Then, the paths from $(v,i)$ to $(v_r,d)$ will be of the form
\[
(v,i) \rightarrow (v_1,i_1) \rightarrow \cdots \rightarrow (v_r,i_r)
\]
where $i \leq i_1 \leq \cdots \leq i_r = d$.
But the number of such choices for $(i_1,\cdots, i_r)$ the same as the number of non-negative integer solutions to $b_1 + \cdots + b_r = d-i$ which is at most $\binom{r + d}{d}$.
We know that the circuit has depth at most $O(\log s)$ and hence $r \leq O(\log s)$.
Therefore, the number of paths from any $(v,i)$ to the root is at most $\binom{d + O(\log s)}{d}$.
Hence, if $\Phi'$ is the formula obtained by unravelling $C$, we have
\[
\mathrm{size}(\Phi') = \poly\inparen{s,\binom{d+\log s}{d}}\qedhere
\]
\end{proof}

\subsubsection*{Set-multilinearization of low-degree formulas}

We shall now show that a homogeneous formula can be converted to a set-multilinear formula with a cost that is affordable if $d$ is small. 

\begin{lemma}[Formula Set-multilinearization]\label{lem:formula set-multilinearization}
  Let $f(\vecx)$ be an $n$-variate degree set-multilinear polynomial with respect to the partition $\vecx = \vecx_1\sqcup \cdots \sqcup \vecx_d$ that is computed by a homogeneous formula $\Phi$ of size $s$.
Then, there exists a set-multilinear formula $\Phi'$ of size at most $2^{O(d\log s)}$ which computes $f$.

In particular, if $d = O\pfrac{\log n}{\log \log n}$ and $s = \poly(n)$ then the $\mathrm{size}(\Phi') = \poly(n)$.  
\end{lemma}
\begin{proof}
To start with, without loss of generality, let us assume that the formula $\Phi$ is fan-in $2$, homogeneous and has depth $O(\log s)$.
In the first step, we set multilinearize $\Phi$ in the obvious way to obtain a circuit $C$.
To this end, for every gate $v$ in $\Phi$, and vector $\veca = (a_1,\cdots, a_d) \in \{0,1\}^d$, there is a gate $(v, \veca)$ in $C$.
Semantically, the the polynomial at $(v,\veca)$ consists of the monomials in the polynomial computed at $v$ (in $\Phi$) which contain exactly one variable from the set $\vecx_i$ for every $i$ such that $a_i = 1$.
The edges in $C$ are connected in a natural way, namely for a gate $v$ with children $u$ and $w$, we have the following edges:
\begin{eqnarray*}
\text{$v = u + w$}\quad\implies\quad (v, \veca) &=& (u, \veca) \spaced{+} (w, \veca)\quad\text{for all $\veca$}\\
\text{$v = u \times w$}\quad\implies\quad (v, \veca) &=& \sum_{\vecb + \vecc = \veca} (u, \vecb)\cdot (w, \vecc) \quad\text{for all $\veca$}.
\end{eqnarray*}
 
Clearly, the size of $C$ is at most $2^d\cdot s$.
Moreover, the gates in $C$ which have out degree more than one are of the form $(u,\veca)$ where $u$ is a child of some multiplication gates at $\Phi$. 
The root of $C$ would now be $(\text{root},\mathbf{1})$. 

Like in the proof of \autoref{lem:formula-homogenization}, we now convert the circuit $C$ to a formula by replicating notes whenever we need to reuse computations.
Hence, we would require as many copies of a gate $(v, \veca)$ as there are paths from $(v,\veca)$ to the root of $C$.
In order to bound the blow up in size in the process, we will prove an upper bound on the number of such paths.
Once again, let $v \rightarrow v_1 \rightarrow \cdots \rightarrow v_r$ be the path from $v$ to the root $(=v_r)$.
Then any path from $(v,\veca)$ to $(v_r,\mathbf{1})$ is of the form
\[
(v,\veca) \rightarrow (v_1,\veca_1) \rightarrow \cdots \rightarrow (v_r,\mathbf{1})
\]
with $\veca \leq \veca_1 \leq \cdots \leq \mathbf{1}$ in the point-wise sense.
Therefore, the number of paths from $(v,\veca)$ to the root is at most the number of sequences of vectors in $2^{dr} = 2^{O(d\log s)}$.
Hence the size of the resulting formula $\Phi'$ is at most $s \cdot 2^{O(d \log s)}$. 
\end{proof}

\autoref{lem:formula-homogenization} and \autoref{lem:formula set-multilinearization} complete the proof of \autoref{thm:form-to-smlform}. 


%%% Local Variables: 
%%% mode: latex
%%% TeX-master: "fancymain"
%%% End: 


